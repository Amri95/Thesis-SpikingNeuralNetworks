
    




    
\documentclass[11pt]{article}

    
    \usepackage[breakable]{tcolorbox}
    \tcbset{nobeforeafter} % prevents tcolorboxes being placing in paragraphs
    \usepackage{float}
    \floatplacement{figure}{H} % forces figures to be placed at the correct location
    
    \usepackage[T1]{fontenc}
    % Nicer default font (+ math font) than Computer Modern for most use cases
    \usepackage{mathpazo}

    % Basic figure setup, for now with no caption control since it's done
    % automatically by Pandoc (which extracts ![](path) syntax from Markdown).
    \usepackage{graphicx}
    % We will generate all images so they have a width \maxwidth. This means
    % that they will get their normal width if they fit onto the page, but
    % are scaled down if they would overflow the margins.
    \makeatletter
    \def\maxwidth{\ifdim\Gin@nat@width>\linewidth\linewidth
    \else\Gin@nat@width\fi}
    \makeatother
    \let\Oldincludegraphics\includegraphics
    % Set max figure width to be 80% of text width, for now hardcoded.
    \renewcommand{\includegraphics}[1]{\Oldincludegraphics[width=.8\maxwidth]{#1}}
    % Ensure that by default, figures have no caption (until we provide a
    % proper Figure object with a Caption API and a way to capture that
    % in the conversion process - todo).
    \usepackage{caption}
    \DeclareCaptionLabelFormat{nolabel}{}
    \captionsetup{labelformat=nolabel}

    \usepackage{adjustbox} % Used to constrain images to a maximum size 
    \usepackage{xcolor} % Allow colors to be defined
    \usepackage{enumerate} % Needed for markdown enumerations to work
    \usepackage{geometry} % Used to adjust the document margins
    \usepackage{amsmath} % Equations
    \usepackage{amssymb} % Equations
    \usepackage{textcomp} % defines textquotesingle
    % Hack from http://tex.stackexchange.com/a/47451/13684:
    \AtBeginDocument{%
        \def\PYZsq{\textquotesingle}% Upright quotes in Pygmentized code
    }
    \usepackage{upquote} % Upright quotes for verbatim code
    \usepackage{eurosym} % defines \euro
    \usepackage[mathletters]{ucs} % Extended unicode (utf-8) support
    \usepackage[utf8x]{inputenc} % Allow utf-8 characters in the tex document
    \usepackage{fancyvrb} % verbatim replacement that allows latex
    \usepackage{grffile} % extends the file name processing of package graphics 
                         % to support a larger range 
    % The hyperref package gives us a pdf with properly built
    % internal navigation ('pdf bookmarks' for the table of contents,
    % internal cross-reference links, web links for URLs, etc.)
    \usepackage{hyperref}
    \usepackage{longtable} % longtable support required by pandoc >1.10
    \usepackage{booktabs}  % table support for pandoc > 1.12.2
    \usepackage[inline]{enumitem} % IRkernel/repr support (it uses the enumerate* environment)
    \usepackage[normalem]{ulem} % ulem is needed to support strikethroughs (\sout)
                                % normalem makes italics be italics, not underlines
    \usepackage{mathrsfs}
    

    
    % Colors for the hyperref package
    \definecolor{urlcolor}{rgb}{0,.145,.698}
    \definecolor{linkcolor}{rgb}{.71,0.21,0.01}
    \definecolor{citecolor}{rgb}{.12,.54,.11}

    % ANSI colors
    \definecolor{ansi-black}{HTML}{3E424D}
    \definecolor{ansi-black-intense}{HTML}{282C36}
    \definecolor{ansi-red}{HTML}{E75C58}
    \definecolor{ansi-red-intense}{HTML}{B22B31}
    \definecolor{ansi-green}{HTML}{00A250}
    \definecolor{ansi-green-intense}{HTML}{007427}
    \definecolor{ansi-yellow}{HTML}{DDB62B}
    \definecolor{ansi-yellow-intense}{HTML}{B27D12}
    \definecolor{ansi-blue}{HTML}{208FFB}
    \definecolor{ansi-blue-intense}{HTML}{0065CA}
    \definecolor{ansi-magenta}{HTML}{D160C4}
    \definecolor{ansi-magenta-intense}{HTML}{A03196}
    \definecolor{ansi-cyan}{HTML}{60C6C8}
    \definecolor{ansi-cyan-intense}{HTML}{258F8F}
    \definecolor{ansi-white}{HTML}{C5C1B4}
    \definecolor{ansi-white-intense}{HTML}{A1A6B2}
    \definecolor{ansi-default-inverse-fg}{HTML}{FFFFFF}
    \definecolor{ansi-default-inverse-bg}{HTML}{000000}

    % commands and environments needed by pandoc snippets
    % extracted from the output of `pandoc -s`
    \providecommand{\tightlist}{%
      \setlength{\itemsep}{0pt}\setlength{\parskip}{0pt}}
    \DefineVerbatimEnvironment{Highlighting}{Verbatim}{commandchars=\\\{\}}
    % Add ',fontsize=\small' for more characters per line
    \newenvironment{Shaded}{}{}
    \newcommand{\KeywordTok}[1]{\textcolor[rgb]{0.00,0.44,0.13}{\textbf{{#1}}}}
    \newcommand{\DataTypeTok}[1]{\textcolor[rgb]{0.56,0.13,0.00}{{#1}}}
    \newcommand{\DecValTok}[1]{\textcolor[rgb]{0.25,0.63,0.44}{{#1}}}
    \newcommand{\BaseNTok}[1]{\textcolor[rgb]{0.25,0.63,0.44}{{#1}}}
    \newcommand{\FloatTok}[1]{\textcolor[rgb]{0.25,0.63,0.44}{{#1}}}
    \newcommand{\CharTok}[1]{\textcolor[rgb]{0.25,0.44,0.63}{{#1}}}
    \newcommand{\StringTok}[1]{\textcolor[rgb]{0.25,0.44,0.63}{{#1}}}
    \newcommand{\CommentTok}[1]{\textcolor[rgb]{0.38,0.63,0.69}{\textit{{#1}}}}
    \newcommand{\OtherTok}[1]{\textcolor[rgb]{0.00,0.44,0.13}{{#1}}}
    \newcommand{\AlertTok}[1]{\textcolor[rgb]{1.00,0.00,0.00}{\textbf{{#1}}}}
    \newcommand{\FunctionTok}[1]{\textcolor[rgb]{0.02,0.16,0.49}{{#1}}}
    \newcommand{\RegionMarkerTok}[1]{{#1}}
    \newcommand{\ErrorTok}[1]{\textcolor[rgb]{1.00,0.00,0.00}{\textbf{{#1}}}}
    \newcommand{\NormalTok}[1]{{#1}}
    
    % Additional commands for more recent versions of Pandoc
    \newcommand{\ConstantTok}[1]{\textcolor[rgb]{0.53,0.00,0.00}{{#1}}}
    \newcommand{\SpecialCharTok}[1]{\textcolor[rgb]{0.25,0.44,0.63}{{#1}}}
    \newcommand{\VerbatimStringTok}[1]{\textcolor[rgb]{0.25,0.44,0.63}{{#1}}}
    \newcommand{\SpecialStringTok}[1]{\textcolor[rgb]{0.73,0.40,0.53}{{#1}}}
    \newcommand{\ImportTok}[1]{{#1}}
    \newcommand{\DocumentationTok}[1]{\textcolor[rgb]{0.73,0.13,0.13}{\textit{{#1}}}}
    \newcommand{\AnnotationTok}[1]{\textcolor[rgb]{0.38,0.63,0.69}{\textbf{\textit{{#1}}}}}
    \newcommand{\CommentVarTok}[1]{\textcolor[rgb]{0.38,0.63,0.69}{\textbf{\textit{{#1}}}}}
    \newcommand{\VariableTok}[1]{\textcolor[rgb]{0.10,0.09,0.49}{{#1}}}
    \newcommand{\ControlFlowTok}[1]{\textcolor[rgb]{0.00,0.44,0.13}{\textbf{{#1}}}}
    \newcommand{\OperatorTok}[1]{\textcolor[rgb]{0.40,0.40,0.40}{{#1}}}
    \newcommand{\BuiltInTok}[1]{{#1}}
    \newcommand{\ExtensionTok}[1]{{#1}}
    \newcommand{\PreprocessorTok}[1]{\textcolor[rgb]{0.74,0.48,0.00}{{#1}}}
    \newcommand{\AttributeTok}[1]{\textcolor[rgb]{0.49,0.56,0.16}{{#1}}}
    \newcommand{\InformationTok}[1]{\textcolor[rgb]{0.38,0.63,0.69}{\textbf{\textit{{#1}}}}}
    \newcommand{\WarningTok}[1]{\textcolor[rgb]{0.38,0.63,0.69}{\textbf{\textit{{#1}}}}}
    
    
    % Define a nice break command that doesn't care if a line doesn't already
    % exist.
    \def\br{\hspace*{\fill} \\* }
    % Math Jax compatibility definitions
    \def\gt{>}
    \def\lt{<}
    \let\Oldtex\TeX
    \let\Oldlatex\LaTeX
    \renewcommand{\TeX}{\textrm{\Oldtex}}
    \renewcommand{\LaTeX}{\textrm{\Oldlatex}}
    % Document parameters
    % Document title
    \title{intro}
    
    
    
    
    
% Pygments definitions
\makeatletter
\def\PY@reset{\let\PY@it=\relax \let\PY@bf=\relax%
    \let\PY@ul=\relax \let\PY@tc=\relax%
    \let\PY@bc=\relax \let\PY@ff=\relax}
\def\PY@tok#1{\csname PY@tok@#1\endcsname}
\def\PY@toks#1+{\ifx\relax#1\empty\else%
    \PY@tok{#1}\expandafter\PY@toks\fi}
\def\PY@do#1{\PY@bc{\PY@tc{\PY@ul{%
    \PY@it{\PY@bf{\PY@ff{#1}}}}}}}
\def\PY#1#2{\PY@reset\PY@toks#1+\relax+\PY@do{#2}}

\expandafter\def\csname PY@tok@w\endcsname{\def\PY@tc##1{\textcolor[rgb]{0.73,0.73,0.73}{##1}}}
\expandafter\def\csname PY@tok@c\endcsname{\let\PY@it=\textit\def\PY@tc##1{\textcolor[rgb]{0.25,0.50,0.50}{##1}}}
\expandafter\def\csname PY@tok@cp\endcsname{\def\PY@tc##1{\textcolor[rgb]{0.74,0.48,0.00}{##1}}}
\expandafter\def\csname PY@tok@k\endcsname{\let\PY@bf=\textbf\def\PY@tc##1{\textcolor[rgb]{0.00,0.50,0.00}{##1}}}
\expandafter\def\csname PY@tok@kp\endcsname{\def\PY@tc##1{\textcolor[rgb]{0.00,0.50,0.00}{##1}}}
\expandafter\def\csname PY@tok@kt\endcsname{\def\PY@tc##1{\textcolor[rgb]{0.69,0.00,0.25}{##1}}}
\expandafter\def\csname PY@tok@o\endcsname{\def\PY@tc##1{\textcolor[rgb]{0.40,0.40,0.40}{##1}}}
\expandafter\def\csname PY@tok@ow\endcsname{\let\PY@bf=\textbf\def\PY@tc##1{\textcolor[rgb]{0.67,0.13,1.00}{##1}}}
\expandafter\def\csname PY@tok@nb\endcsname{\def\PY@tc##1{\textcolor[rgb]{0.00,0.50,0.00}{##1}}}
\expandafter\def\csname PY@tok@nf\endcsname{\def\PY@tc##1{\textcolor[rgb]{0.00,0.00,1.00}{##1}}}
\expandafter\def\csname PY@tok@nc\endcsname{\let\PY@bf=\textbf\def\PY@tc##1{\textcolor[rgb]{0.00,0.00,1.00}{##1}}}
\expandafter\def\csname PY@tok@nn\endcsname{\let\PY@bf=\textbf\def\PY@tc##1{\textcolor[rgb]{0.00,0.00,1.00}{##1}}}
\expandafter\def\csname PY@tok@ne\endcsname{\let\PY@bf=\textbf\def\PY@tc##1{\textcolor[rgb]{0.82,0.25,0.23}{##1}}}
\expandafter\def\csname PY@tok@nv\endcsname{\def\PY@tc##1{\textcolor[rgb]{0.10,0.09,0.49}{##1}}}
\expandafter\def\csname PY@tok@no\endcsname{\def\PY@tc##1{\textcolor[rgb]{0.53,0.00,0.00}{##1}}}
\expandafter\def\csname PY@tok@nl\endcsname{\def\PY@tc##1{\textcolor[rgb]{0.63,0.63,0.00}{##1}}}
\expandafter\def\csname PY@tok@ni\endcsname{\let\PY@bf=\textbf\def\PY@tc##1{\textcolor[rgb]{0.60,0.60,0.60}{##1}}}
\expandafter\def\csname PY@tok@na\endcsname{\def\PY@tc##1{\textcolor[rgb]{0.49,0.56,0.16}{##1}}}
\expandafter\def\csname PY@tok@nt\endcsname{\let\PY@bf=\textbf\def\PY@tc##1{\textcolor[rgb]{0.00,0.50,0.00}{##1}}}
\expandafter\def\csname PY@tok@nd\endcsname{\def\PY@tc##1{\textcolor[rgb]{0.67,0.13,1.00}{##1}}}
\expandafter\def\csname PY@tok@s\endcsname{\def\PY@tc##1{\textcolor[rgb]{0.73,0.13,0.13}{##1}}}
\expandafter\def\csname PY@tok@sd\endcsname{\let\PY@it=\textit\def\PY@tc##1{\textcolor[rgb]{0.73,0.13,0.13}{##1}}}
\expandafter\def\csname PY@tok@si\endcsname{\let\PY@bf=\textbf\def\PY@tc##1{\textcolor[rgb]{0.73,0.40,0.53}{##1}}}
\expandafter\def\csname PY@tok@se\endcsname{\let\PY@bf=\textbf\def\PY@tc##1{\textcolor[rgb]{0.73,0.40,0.13}{##1}}}
\expandafter\def\csname PY@tok@sr\endcsname{\def\PY@tc##1{\textcolor[rgb]{0.73,0.40,0.53}{##1}}}
\expandafter\def\csname PY@tok@ss\endcsname{\def\PY@tc##1{\textcolor[rgb]{0.10,0.09,0.49}{##1}}}
\expandafter\def\csname PY@tok@sx\endcsname{\def\PY@tc##1{\textcolor[rgb]{0.00,0.50,0.00}{##1}}}
\expandafter\def\csname PY@tok@m\endcsname{\def\PY@tc##1{\textcolor[rgb]{0.40,0.40,0.40}{##1}}}
\expandafter\def\csname PY@tok@gh\endcsname{\let\PY@bf=\textbf\def\PY@tc##1{\textcolor[rgb]{0.00,0.00,0.50}{##1}}}
\expandafter\def\csname PY@tok@gu\endcsname{\let\PY@bf=\textbf\def\PY@tc##1{\textcolor[rgb]{0.50,0.00,0.50}{##1}}}
\expandafter\def\csname PY@tok@gd\endcsname{\def\PY@tc##1{\textcolor[rgb]{0.63,0.00,0.00}{##1}}}
\expandafter\def\csname PY@tok@gi\endcsname{\def\PY@tc##1{\textcolor[rgb]{0.00,0.63,0.00}{##1}}}
\expandafter\def\csname PY@tok@gr\endcsname{\def\PY@tc##1{\textcolor[rgb]{1.00,0.00,0.00}{##1}}}
\expandafter\def\csname PY@tok@ge\endcsname{\let\PY@it=\textit}
\expandafter\def\csname PY@tok@gs\endcsname{\let\PY@bf=\textbf}
\expandafter\def\csname PY@tok@gp\endcsname{\let\PY@bf=\textbf\def\PY@tc##1{\textcolor[rgb]{0.00,0.00,0.50}{##1}}}
\expandafter\def\csname PY@tok@go\endcsname{\def\PY@tc##1{\textcolor[rgb]{0.53,0.53,0.53}{##1}}}
\expandafter\def\csname PY@tok@gt\endcsname{\def\PY@tc##1{\textcolor[rgb]{0.00,0.27,0.87}{##1}}}
\expandafter\def\csname PY@tok@err\endcsname{\def\PY@bc##1{\setlength{\fboxsep}{0pt}\fcolorbox[rgb]{1.00,0.00,0.00}{1,1,1}{\strut ##1}}}
\expandafter\def\csname PY@tok@kc\endcsname{\let\PY@bf=\textbf\def\PY@tc##1{\textcolor[rgb]{0.00,0.50,0.00}{##1}}}
\expandafter\def\csname PY@tok@kd\endcsname{\let\PY@bf=\textbf\def\PY@tc##1{\textcolor[rgb]{0.00,0.50,0.00}{##1}}}
\expandafter\def\csname PY@tok@kn\endcsname{\let\PY@bf=\textbf\def\PY@tc##1{\textcolor[rgb]{0.00,0.50,0.00}{##1}}}
\expandafter\def\csname PY@tok@kr\endcsname{\let\PY@bf=\textbf\def\PY@tc##1{\textcolor[rgb]{0.00,0.50,0.00}{##1}}}
\expandafter\def\csname PY@tok@bp\endcsname{\def\PY@tc##1{\textcolor[rgb]{0.00,0.50,0.00}{##1}}}
\expandafter\def\csname PY@tok@fm\endcsname{\def\PY@tc##1{\textcolor[rgb]{0.00,0.00,1.00}{##1}}}
\expandafter\def\csname PY@tok@vc\endcsname{\def\PY@tc##1{\textcolor[rgb]{0.10,0.09,0.49}{##1}}}
\expandafter\def\csname PY@tok@vg\endcsname{\def\PY@tc##1{\textcolor[rgb]{0.10,0.09,0.49}{##1}}}
\expandafter\def\csname PY@tok@vi\endcsname{\def\PY@tc##1{\textcolor[rgb]{0.10,0.09,0.49}{##1}}}
\expandafter\def\csname PY@tok@vm\endcsname{\def\PY@tc##1{\textcolor[rgb]{0.10,0.09,0.49}{##1}}}
\expandafter\def\csname PY@tok@sa\endcsname{\def\PY@tc##1{\textcolor[rgb]{0.73,0.13,0.13}{##1}}}
\expandafter\def\csname PY@tok@sb\endcsname{\def\PY@tc##1{\textcolor[rgb]{0.73,0.13,0.13}{##1}}}
\expandafter\def\csname PY@tok@sc\endcsname{\def\PY@tc##1{\textcolor[rgb]{0.73,0.13,0.13}{##1}}}
\expandafter\def\csname PY@tok@dl\endcsname{\def\PY@tc##1{\textcolor[rgb]{0.73,0.13,0.13}{##1}}}
\expandafter\def\csname PY@tok@s2\endcsname{\def\PY@tc##1{\textcolor[rgb]{0.73,0.13,0.13}{##1}}}
\expandafter\def\csname PY@tok@sh\endcsname{\def\PY@tc##1{\textcolor[rgb]{0.73,0.13,0.13}{##1}}}
\expandafter\def\csname PY@tok@s1\endcsname{\def\PY@tc##1{\textcolor[rgb]{0.73,0.13,0.13}{##1}}}
\expandafter\def\csname PY@tok@mb\endcsname{\def\PY@tc##1{\textcolor[rgb]{0.40,0.40,0.40}{##1}}}
\expandafter\def\csname PY@tok@mf\endcsname{\def\PY@tc##1{\textcolor[rgb]{0.40,0.40,0.40}{##1}}}
\expandafter\def\csname PY@tok@mh\endcsname{\def\PY@tc##1{\textcolor[rgb]{0.40,0.40,0.40}{##1}}}
\expandafter\def\csname PY@tok@mi\endcsname{\def\PY@tc##1{\textcolor[rgb]{0.40,0.40,0.40}{##1}}}
\expandafter\def\csname PY@tok@il\endcsname{\def\PY@tc##1{\textcolor[rgb]{0.40,0.40,0.40}{##1}}}
\expandafter\def\csname PY@tok@mo\endcsname{\def\PY@tc##1{\textcolor[rgb]{0.40,0.40,0.40}{##1}}}
\expandafter\def\csname PY@tok@ch\endcsname{\let\PY@it=\textit\def\PY@tc##1{\textcolor[rgb]{0.25,0.50,0.50}{##1}}}
\expandafter\def\csname PY@tok@cm\endcsname{\let\PY@it=\textit\def\PY@tc##1{\textcolor[rgb]{0.25,0.50,0.50}{##1}}}
\expandafter\def\csname PY@tok@cpf\endcsname{\let\PY@it=\textit\def\PY@tc##1{\textcolor[rgb]{0.25,0.50,0.50}{##1}}}
\expandafter\def\csname PY@tok@c1\endcsname{\let\PY@it=\textit\def\PY@tc##1{\textcolor[rgb]{0.25,0.50,0.50}{##1}}}
\expandafter\def\csname PY@tok@cs\endcsname{\let\PY@it=\textit\def\PY@tc##1{\textcolor[rgb]{0.25,0.50,0.50}{##1}}}

\def\PYZbs{\char`\\}
\def\PYZus{\char`\_}
\def\PYZob{\char`\{}
\def\PYZcb{\char`\}}
\def\PYZca{\char`\^}
\def\PYZam{\char`\&}
\def\PYZlt{\char`\<}
\def\PYZgt{\char`\>}
\def\PYZsh{\char`\#}
\def\PYZpc{\char`\%}
\def\PYZdl{\char`\$}
\def\PYZhy{\char`\-}
\def\PYZsq{\char`\'}
\def\PYZdq{\char`\"}
\def\PYZti{\char`\~}
% for compatibility with earlier versions
\def\PYZat{@}
\def\PYZlb{[}
\def\PYZrb{]}
\makeatother


    % For linebreaks inside Verbatim environment from package fancyvrb. 
    \makeatletter
        \newbox\Wrappedcontinuationbox 
        \newbox\Wrappedvisiblespacebox 
        \newcommand*\Wrappedvisiblespace {\textcolor{red}{\textvisiblespace}} 
        \newcommand*\Wrappedcontinuationsymbol {\textcolor{red}{\llap{\tiny$\m@th\hookrightarrow$}}} 
        \newcommand*\Wrappedcontinuationindent {3ex } 
        \newcommand*\Wrappedafterbreak {\kern\Wrappedcontinuationindent\copy\Wrappedcontinuationbox} 
        % Take advantage of the already applied Pygments mark-up to insert 
        % potential linebreaks for TeX processing. 
        %        {, <, #, %, $, ' and ": go to next line. 
        %        _, }, ^, &, >, - and ~: stay at end of broken line. 
        % Use of \textquotesingle for straight quote. 
        \newcommand*\Wrappedbreaksatspecials {% 
            \def\PYGZus{\discretionary{\char`\_}{\Wrappedafterbreak}{\char`\_}}% 
            \def\PYGZob{\discretionary{}{\Wrappedafterbreak\char`\{}{\char`\{}}% 
            \def\PYGZcb{\discretionary{\char`\}}{\Wrappedafterbreak}{\char`\}}}% 
            \def\PYGZca{\discretionary{\char`\^}{\Wrappedafterbreak}{\char`\^}}% 
            \def\PYGZam{\discretionary{\char`\&}{\Wrappedafterbreak}{\char`\&}}% 
            \def\PYGZlt{\discretionary{}{\Wrappedafterbreak\char`\<}{\char`\<}}% 
            \def\PYGZgt{\discretionary{\char`\>}{\Wrappedafterbreak}{\char`\>}}% 
            \def\PYGZsh{\discretionary{}{\Wrappedafterbreak\char`\#}{\char`\#}}% 
            \def\PYGZpc{\discretionary{}{\Wrappedafterbreak\char`\%}{\char`\%}}% 
            \def\PYGZdl{\discretionary{}{\Wrappedafterbreak\char`\$}{\char`\$}}% 
            \def\PYGZhy{\discretionary{\char`\-}{\Wrappedafterbreak}{\char`\-}}% 
            \def\PYGZsq{\discretionary{}{\Wrappedafterbreak\textquotesingle}{\textquotesingle}}% 
            \def\PYGZdq{\discretionary{}{\Wrappedafterbreak\char`\"}{\char`\"}}% 
            \def\PYGZti{\discretionary{\char`\~}{\Wrappedafterbreak}{\char`\~}}% 
        } 
        % Some characters . , ; ? ! / are not pygmentized. 
        % This macro makes them "active" and they will insert potential linebreaks 
        \newcommand*\Wrappedbreaksatpunct {% 
            \lccode`\~`\.\lowercase{\def~}{\discretionary{\hbox{\char`\.}}{\Wrappedafterbreak}{\hbox{\char`\.}}}% 
            \lccode`\~`\,\lowercase{\def~}{\discretionary{\hbox{\char`\,}}{\Wrappedafterbreak}{\hbox{\char`\,}}}% 
            \lccode`\~`\;\lowercase{\def~}{\discretionary{\hbox{\char`\;}}{\Wrappedafterbreak}{\hbox{\char`\;}}}% 
            \lccode`\~`\:\lowercase{\def~}{\discretionary{\hbox{\char`\:}}{\Wrappedafterbreak}{\hbox{\char`\:}}}% 
            \lccode`\~`\?\lowercase{\def~}{\discretionary{\hbox{\char`\?}}{\Wrappedafterbreak}{\hbox{\char`\?}}}% 
            \lccode`\~`\!\lowercase{\def~}{\discretionary{\hbox{\char`\!}}{\Wrappedafterbreak}{\hbox{\char`\!}}}% 
            \lccode`\~`\/\lowercase{\def~}{\discretionary{\hbox{\char`\/}}{\Wrappedafterbreak}{\hbox{\char`\/}}}% 
            \catcode`\.\active
            \catcode`\,\active 
            \catcode`\;\active
            \catcode`\:\active
            \catcode`\?\active
            \catcode`\!\active
            \catcode`\/\active 
            \lccode`\~`\~ 	
        }
    \makeatother

    \let\OriginalVerbatim=\Verbatim
    \makeatletter
    \renewcommand{\Verbatim}[1][1]{%
        %\parskip\z@skip
        \sbox\Wrappedcontinuationbox {\Wrappedcontinuationsymbol}%
        \sbox\Wrappedvisiblespacebox {\FV@SetupFont\Wrappedvisiblespace}%
        \def\FancyVerbFormatLine ##1{\hsize\linewidth
            \vtop{\raggedright\hyphenpenalty\z@\exhyphenpenalty\z@
                \doublehyphendemerits\z@\finalhyphendemerits\z@
                \strut ##1\strut}%
        }%
        % If the linebreak is at a space, the latter will be displayed as visible
        % space at end of first line, and a continuation symbol starts next line.
        % Stretch/shrink are however usually zero for typewriter font.
        \def\FV@Space {%
            \nobreak\hskip\z@ plus\fontdimen3\font minus\fontdimen4\font
            \discretionary{\copy\Wrappedvisiblespacebox}{\Wrappedafterbreak}
            {\kern\fontdimen2\font}%
        }%
        
        % Allow breaks at special characters using \PYG... macros.
        \Wrappedbreaksatspecials
        % Breaks at punctuation characters . , ; ? ! and / need catcode=\active 	
        \OriginalVerbatim[#1,codes*=\Wrappedbreaksatpunct]%
    }
    \makeatother

    % Exact colors from NB
    \definecolor{incolor}{HTML}{303F9F}
    \definecolor{outcolor}{HTML}{D84315}
    \definecolor{cellborder}{HTML}{CFCFCF}
    \definecolor{cellbackground}{HTML}{F7F7F7}
    
    % prompt
    \newcommand{\prompt}[4]{
        \llap{{\color{#2}[#3]: #4}}\vspace{-1.25em}
    }
    

    
    % Prevent overflowing lines due to hard-to-break entities
    \sloppy 
    % Setup hyperref package
    \hypersetup{
      breaklinks=true,  % so long urls are correctly broken across lines
      colorlinks=true,
      urlcolor=urlcolor,
      linkcolor=linkcolor,
      citecolor=citecolor,
      }
    % Slightly bigger margins than the latex defaults
    
    \geometry{verbose,tmargin=1in,bmargin=1in,lmargin=1in,rmargin=1in}
    
    

    \begin{document}
    
    
    \maketitle
    
    

    
    Nengo is based in three main cornerstones: * Representation *
Transformation * Dynamics

These three principles are going to be explained in this notebook along
with the basic components of Nengo.

    \begin{tcolorbox}[breakable, size=fbox, boxrule=1pt, pad at break*=1mm,colback=cellbackground, colframe=cellborder]
\prompt{In}{incolor}{1}{\hspace{4pt}}
\begin{Verbatim}[commandchars=\\\{\}]
\PY{o}{\PYZpc{}}\PY{k}{matplotlib} inline
\PY{k+kn}{import} \PY{n+nn}{numpy} \PY{k}{as} \PY{n+nn}{np}
\PY{k+kn}{import} \PY{n+nn}{matplotlib}\PY{n+nn}{.}\PY{n+nn}{pyplot} \PY{k}{as} \PY{n+nn}{plt}

\PY{k+kn}{import} \PY{n+nn}{nengo}
\PY{k+kn}{from} \PY{n+nn}{nengo}\PY{n+nn}{.}\PY{n+nn}{dists} \PY{k}{import} \PY{n}{Uniform}
\PY{k+kn}{from} \PY{n+nn}{nengo}\PY{n+nn}{.}\PY{n+nn}{processes} \PY{k}{import} \PY{n}{WhiteSignal}
\PY{k+kn}{from} \PY{n+nn}{nengo}\PY{n+nn}{.}\PY{n+nn}{utils}\PY{n+nn}{.}\PY{n+nn}{ensemble} \PY{k}{import} \PY{n}{tuning\PYZus{}curves}
\PY{k+kn}{from} \PY{n+nn}{nengo}\PY{n+nn}{.}\PY{n+nn}{utils}\PY{n+nn}{.}\PY{n+nn}{ipython} \PY{k}{import} \PY{n}{hide\PYZus{}input}
\PY{k+kn}{from} \PY{n+nn}{nengo}\PY{n+nn}{.}\PY{n+nn}{utils}\PY{n+nn}{.}\PY{n+nn}{matplotlib} \PY{k}{import} \PY{n}{rasterplot}


\PY{k}{def} \PY{n+nf}{aligned}\PY{p}{(}\PY{n}{n\PYZus{}neurons}\PY{p}{,} \PY{n}{radius}\PY{o}{=}\PY{l+m+mf}{0.9}\PY{p}{)}\PY{p}{:}
    \PY{n}{intercepts} \PY{o}{=} \PY{n}{np}\PY{o}{.}\PY{n}{linspace}\PY{p}{(}\PY{o}{\PYZhy{}}\PY{n}{radius}\PY{p}{,} \PY{n}{radius}\PY{p}{,} \PY{n}{n\PYZus{}neurons}\PY{p}{)}
    \PY{n}{encoders} \PY{o}{=} \PY{n}{np}\PY{o}{.}\PY{n}{tile}\PY{p}{(}\PY{p}{[}\PY{p}{[}\PY{l+m+mi}{1}\PY{p}{]}\PY{p}{,} \PY{p}{[}\PY{o}{\PYZhy{}}\PY{l+m+mi}{1}\PY{p}{]}\PY{p}{]}\PY{p}{,} \PY{p}{(}\PY{n}{n\PYZus{}neurons} \PY{o}{/}\PY{o}{/} \PY{l+m+mi}{2}\PY{p}{,} \PY{l+m+mi}{1}\PY{p}{)}\PY{p}{)}
    \PY{n}{intercepts} \PY{o}{*}\PY{o}{=} \PY{n}{encoders}\PY{p}{[}\PY{p}{:}\PY{p}{,} \PY{l+m+mi}{0}\PY{p}{]}
    \PY{k}{return} \PY{n}{intercepts}\PY{p}{,} \PY{n}{encoders}


\PY{n}{hide\PYZus{}input}\PY{p}{(}\PY{p}{)}
\end{Verbatim}
\end{tcolorbox}

            \begin{tcolorbox}[breakable, boxrule=.5pt, size=fbox, pad at break*=1mm, opacityfill=0]
\prompt{Out}{outcolor}{1}{\hspace{3.5pt}}
\begin{Verbatim}[commandchars=\\\{\}]
<IPython.core.display.HTML object>
\end{Verbatim}
\end{tcolorbox}
        
    \hypertarget{principle-1-representation}{%
\subsection{Principle 1:
Representation}\label{principle-1-representation}}

\hypertarget{encoding}{%
\subsubsection{Encoding}\label{encoding}}

Neural populations represent time-varying signals through their spiking
responses. A signal is a vector of real numbers of arbitrary length.
This example is a 1D signal going from -1 to 1 in 1 second.

    \begin{tcolorbox}[breakable, size=fbox, boxrule=1pt, pad at break*=1mm,colback=cellbackground, colframe=cellborder]
\prompt{In}{incolor}{2}{\hspace{4pt}}
\begin{Verbatim}[commandchars=\\\{\}]
\PY{n}{model} \PY{o}{=} \PY{n}{nengo}\PY{o}{.}\PY{n}{Network}\PY{p}{(}\PY{n}{label}\PY{o}{=}\PY{l+s+s2}{\PYZdq{}}\PY{l+s+s2}{NET}\PY{l+s+s2}{\PYZdq{}}\PY{p}{)}
\PY{k}{with} \PY{n}{model}\PY{p}{:}
    \PY{n+nb}{input} \PY{o}{=} \PY{n}{nengo}\PY{o}{.}\PY{n}{Node}\PY{p}{(}\PY{k}{lambda} \PY{n}{t}\PY{p}{:} \PY{n}{t} \PY{o}{*} \PY{l+m+mi}{2} \PY{o}{\PYZhy{}} \PY{l+m+mi}{1}\PY{p}{)}
    \PY{n}{input\PYZus{}probe} \PY{o}{=} \PY{n}{nengo}\PY{o}{.}\PY{n}{Probe}\PY{p}{(}\PY{n+nb}{input}\PY{p}{)}
\end{Verbatim}
\end{tcolorbox}

    A class Network from nengo is created using the first line in the above
cell.

    \begin{tcolorbox}[breakable, size=fbox, boxrule=1pt, pad at break*=1mm,colback=cellbackground, colframe=cellborder]
\prompt{In}{incolor}{3}{\hspace{4pt}}
\begin{Verbatim}[commandchars=\\\{\}]
\PY{k}{with} \PY{n}{nengo}\PY{o}{.}\PY{n}{Simulator}\PY{p}{(}\PY{n}{model}\PY{p}{)} \PY{k}{as} \PY{n}{sim}\PY{p}{:}
    \PY{n}{sim}\PY{o}{.}\PY{n}{run}\PY{p}{(}\PY{l+m+mf}{1.0}\PY{p}{)}

\PY{n}{plt}\PY{o}{.}\PY{n}{figure}\PY{p}{(}\PY{p}{)}
\PY{n}{plt}\PY{o}{.}\PY{n}{plot}\PY{p}{(}\PY{n}{sim}\PY{o}{.}\PY{n}{trange}\PY{p}{(}\PY{p}{)}\PY{p}{,} \PY{n}{sim}\PY{o}{.}\PY{n}{data}\PY{p}{[}\PY{n}{input\PYZus{}probe}\PY{p}{]}\PY{p}{,} \PY{n}{lw}\PY{o}{=}\PY{l+m+mi}{2}\PY{p}{)}
\PY{n}{plt}\PY{o}{.}\PY{n}{title}\PY{p}{(}\PY{l+s+s2}{\PYZdq{}}\PY{l+s+s2}{Input signal}\PY{l+s+s2}{\PYZdq{}}\PY{p}{)}
\PY{n}{plt}\PY{o}{.}\PY{n}{xlabel}\PY{p}{(}\PY{l+s+s2}{\PYZdq{}}\PY{l+s+s2}{Time (s)}\PY{l+s+s2}{\PYZdq{}}\PY{p}{)}
\PY{n}{plt}\PY{o}{.}\PY{n}{xlim}\PY{p}{(}\PY{l+m+mi}{0}\PY{p}{,} \PY{l+m+mi}{1}\PY{p}{)}\PY{p}{;}
\PY{n}{hide\PYZus{}input}\PY{p}{(}\PY{p}{)}
\end{Verbatim}
\end{tcolorbox}

    
    \begin{verbatim}
HtmlProgressBar cannot be displayed. Please use the TerminalProgressBar. It can be enabled with `nengo.rc.set('progress', 'progress_bar', 'nengo.utils.progress.TerminalProgressBar')`.
    \end{verbatim}

    
    
    
    
    \begin{verbatim}
HtmlProgressBar cannot be displayed. Please use the TerminalProgressBar. It can be enabled with `nengo.rc.set('progress', 'progress_bar', 'nengo.utils.progress.TerminalProgressBar')`.
    \end{verbatim}

    
    
    
            \begin{tcolorbox}[breakable, boxrule=.5pt, size=fbox, pad at break*=1mm, opacityfill=0]
\prompt{Out}{outcolor}{3}{\hspace{3.5pt}}
\begin{Verbatim}[commandchars=\\\{\}]
<IPython.core.display.HTML object>
\end{Verbatim}
\end{tcolorbox}
        
    \begin{center}
    \adjustimage{max size={0.9\linewidth}{0.9\paperheight}}{intro_files/intro_5_5.png}
    \end{center}
    { \hspace*{\fill} \\}
    
    These signals drive neural populations based on each neuron's
\emph{tuning curve} (which is similar to the current-frequency curve, if
you're familiar with that).

The tuning curve describes how much a particular neuron will fire as a
function of the input signal.

    \begin{tcolorbox}[breakable, size=fbox, boxrule=1pt, pad at break*=1mm,colback=cellbackground, colframe=cellborder]
\prompt{In}{incolor}{4}{\hspace{4pt}}
\begin{Verbatim}[commandchars=\\\{\}]
\PY{n}{intercepts}\PY{p}{,} \PY{n}{encoders} \PY{o}{=} \PY{n}{aligned}\PY{p}{(}\PY{l+m+mi}{8}\PY{p}{)}  \PY{c+c1}{\PYZsh{} Makes evenly spaced intercepts}
\PY{k}{with} \PY{n}{model}\PY{p}{:}
    \PY{n}{A} \PY{o}{=} \PY{n}{nengo}\PY{o}{.}\PY{n}{Ensemble}\PY{p}{(}
        \PY{l+m+mi}{8}\PY{p}{,}
        \PY{n}{dimensions}\PY{o}{=}\PY{l+m+mi}{1}\PY{p}{,}
        \PY{n}{intercepts}\PY{o}{=}\PY{n}{intercepts}\PY{p}{,}
        \PY{n}{max\PYZus{}rates}\PY{o}{=}\PY{n}{Uniform}\PY{p}{(}\PY{l+m+mi}{80}\PY{p}{,} \PY{l+m+mi}{100}\PY{p}{)}\PY{p}{,}
        \PY{n}{encoders}\PY{o}{=}\PY{n}{encoders}\PY{p}{)}
\end{Verbatim}
\end{tcolorbox}

    \begin{tcolorbox}[breakable, size=fbox, boxrule=1pt, pad at break*=1mm,colback=cellbackground, colframe=cellborder]
\prompt{In}{incolor}{5}{\hspace{4pt}}
\begin{Verbatim}[commandchars=\\\{\}]
\PY{k}{with} \PY{n}{nengo}\PY{o}{.}\PY{n}{Simulator}\PY{p}{(}\PY{n}{model}\PY{p}{)} \PY{k}{as} \PY{n}{sim}\PY{p}{:}
    \PY{n}{eval\PYZus{}points}\PY{p}{,} \PY{n}{activities} \PY{o}{=} \PY{n}{tuning\PYZus{}curves}\PY{p}{(}\PY{n}{A}\PY{p}{,} \PY{n}{sim}\PY{p}{)}

\PY{n}{plt}\PY{o}{.}\PY{n}{figure}\PY{p}{(}\PY{p}{)}
\PY{n}{plt}\PY{o}{.}\PY{n}{plot}\PY{p}{(}\PY{n}{eval\PYZus{}points}\PY{p}{,} \PY{n}{activities}\PY{p}{,} \PY{n}{lw}\PY{o}{=}\PY{l+m+mi}{2}\PY{p}{)}
\PY{n}{plt}\PY{o}{.}\PY{n}{xlabel}\PY{p}{(}\PY{l+s+s2}{\PYZdq{}}\PY{l+s+s2}{Input signal}\PY{l+s+s2}{\PYZdq{}}\PY{p}{)}
\PY{n}{plt}\PY{o}{.}\PY{n}{ylabel}\PY{p}{(}\PY{l+s+s2}{\PYZdq{}}\PY{l+s+s2}{Firing rate (Hz)}\PY{l+s+s2}{\PYZdq{}}\PY{p}{)}\PY{p}{;}
\PY{n}{hide\PYZus{}input}\PY{p}{(}\PY{p}{)}
\end{Verbatim}
\end{tcolorbox}

    
    \begin{verbatim}
HtmlProgressBar cannot be displayed. Please use the TerminalProgressBar. It can be enabled with `nengo.rc.set('progress', 'progress_bar', 'nengo.utils.progress.TerminalProgressBar')`.
    \end{verbatim}

    
    
    
            \begin{tcolorbox}[breakable, boxrule=.5pt, size=fbox, pad at break*=1mm, opacityfill=0]
\prompt{Out}{outcolor}{5}{\hspace{3.5pt}}
\begin{Verbatim}[commandchars=\\\{\}]
<IPython.core.display.HTML object>
\end{Verbatim}
\end{tcolorbox}
        
    \begin{center}
    \adjustimage{max size={0.9\linewidth}{0.9\paperheight}}{intro_files/intro_8_3.png}
    \end{center}
    { \hspace*{\fill} \\}
    
    We can drive these neurons with our input signal and observe their
spiking activity over time.

    \begin{tcolorbox}[breakable, size=fbox, boxrule=1pt, pad at break*=1mm,colback=cellbackground, colframe=cellborder]
\prompt{In}{incolor}{6}{\hspace{4pt}}
\begin{Verbatim}[commandchars=\\\{\}]
\PY{k}{with} \PY{n}{model}\PY{p}{:}
    \PY{n}{nengo}\PY{o}{.}\PY{n}{Connection}\PY{p}{(}\PY{n+nb}{input}\PY{p}{,} \PY{n}{A}\PY{p}{)}
    \PY{n}{A\PYZus{}spikes} \PY{o}{=} \PY{n}{nengo}\PY{o}{.}\PY{n}{Probe}\PY{p}{(}\PY{n}{A}\PY{o}{.}\PY{n}{neurons}\PY{p}{)}
\end{Verbatim}
\end{tcolorbox}

    \begin{tcolorbox}[breakable, size=fbox, boxrule=1pt, pad at break*=1mm,colback=cellbackground, colframe=cellborder]
\prompt{In}{incolor}{7}{\hspace{4pt}}
\begin{Verbatim}[commandchars=\\\{\}]
\PY{k}{with} \PY{n}{nengo}\PY{o}{.}\PY{n}{Simulator}\PY{p}{(}\PY{n}{model}\PY{p}{)} \PY{k}{as} \PY{n}{sim}\PY{p}{:}
    \PY{n}{sim}\PY{o}{.}\PY{n}{run}\PY{p}{(}\PY{l+m+mi}{1}\PY{p}{)}

\PY{n}{plt}\PY{o}{.}\PY{n}{figure}\PY{p}{(}\PY{p}{)}
\PY{n}{ax} \PY{o}{=} \PY{n}{plt}\PY{o}{.}\PY{n}{subplot}\PY{p}{(}\PY{l+m+mi}{1}\PY{p}{,} \PY{l+m+mi}{1}\PY{p}{,} \PY{l+m+mi}{1}\PY{p}{)}
\PY{n}{rasterplot}\PY{p}{(}\PY{n}{sim}\PY{o}{.}\PY{n}{trange}\PY{p}{(}\PY{p}{)}\PY{p}{,} \PY{n}{sim}\PY{o}{.}\PY{n}{data}\PY{p}{[}\PY{n}{A\PYZus{}spikes}\PY{p}{]}\PY{p}{,} \PY{n}{ax}\PY{p}{)}
\PY{n}{ax}\PY{o}{.}\PY{n}{set\PYZus{}xlim}\PY{p}{(}\PY{l+m+mi}{0}\PY{p}{,} \PY{l+m+mi}{1}\PY{p}{)}
\PY{n}{ax}\PY{o}{.}\PY{n}{set\PYZus{}ylabel}\PY{p}{(}\PY{l+s+s1}{\PYZsq{}}\PY{l+s+s1}{Neuron}\PY{l+s+s1}{\PYZsq{}}\PY{p}{)}
\PY{n}{ax}\PY{o}{.}\PY{n}{set\PYZus{}xlabel}\PY{p}{(}\PY{l+s+s1}{\PYZsq{}}\PY{l+s+s1}{Time (s)}\PY{l+s+s1}{\PYZsq{}}\PY{p}{)}\PY{p}{;}
\PY{n}{hide\PYZus{}input}\PY{p}{(}\PY{p}{)}
\end{Verbatim}
\end{tcolorbox}

    
    \begin{verbatim}
HtmlProgressBar cannot be displayed. Please use the TerminalProgressBar. It can be enabled with `nengo.rc.set('progress', 'progress_bar', 'nengo.utils.progress.TerminalProgressBar')`.
    \end{verbatim}

    
    
    
    
    \begin{verbatim}
HtmlProgressBar cannot be displayed. Please use the TerminalProgressBar. It can be enabled with `nengo.rc.set('progress', 'progress_bar', 'nengo.utils.progress.TerminalProgressBar')`.
    \end{verbatim}

    
    
    
            \begin{tcolorbox}[breakable, boxrule=.5pt, size=fbox, pad at break*=1mm, opacityfill=0]
\prompt{Out}{outcolor}{7}{\hspace{3.5pt}}
\begin{Verbatim}[commandchars=\\\{\}]
<IPython.core.display.HTML object>
\end{Verbatim}
\end{tcolorbox}
        
    \begin{center}
    \adjustimage{max size={0.9\linewidth}{0.9\paperheight}}{intro_files/intro_11_5.png}
    \end{center}
    { \hspace*{\fill} \\}
    
    \hypertarget{decoding}{%
\subsubsection{Decoding}\label{decoding}}

We can estimate the input signal originally encoded by decoding the
pattern of spikes. To do this, we first filter the spike train with a
temporal filter that accounts for postsynaptic current (PSC) activity.

    \begin{tcolorbox}[breakable, size=fbox, boxrule=1pt, pad at break*=1mm,colback=cellbackground, colframe=cellborder]
\prompt{In}{incolor}{8}{\hspace{4pt}}
\begin{Verbatim}[commandchars=\\\{\}]
\PY{n}{model} \PY{o}{=} \PY{n}{nengo}\PY{o}{.}\PY{n}{Network}\PY{p}{(}\PY{n}{label}\PY{o}{=}\PY{l+s+s2}{\PYZdq{}}\PY{l+s+s2}{NEF summary}\PY{l+s+s2}{\PYZdq{}}\PY{p}{)}
\PY{k}{with} \PY{n}{model}\PY{p}{:}
    \PY{n+nb}{input} \PY{o}{=} \PY{n}{nengo}\PY{o}{.}\PY{n}{Node}\PY{p}{(}\PY{k}{lambda} \PY{n}{t}\PY{p}{:} \PY{n}{t} \PY{o}{*} \PY{l+m+mi}{2} \PY{o}{\PYZhy{}} \PY{l+m+mi}{1}\PY{p}{)}
    \PY{n}{input\PYZus{}probe} \PY{o}{=} \PY{n}{nengo}\PY{o}{.}\PY{n}{Probe}\PY{p}{(}\PY{n+nb}{input}\PY{p}{)}
    \PY{n}{intercepts}\PY{p}{,} \PY{n}{encoders} \PY{o}{=} \PY{n}{aligned}\PY{p}{(}\PY{l+m+mi}{8}\PY{p}{)}  \PY{c+c1}{\PYZsh{} Makes evenly spaced intercepts}
    \PY{n}{A} \PY{o}{=} \PY{n}{nengo}\PY{o}{.}\PY{n}{Ensemble}\PY{p}{(}\PY{l+m+mi}{8}\PY{p}{,} \PY{n}{dimensions}\PY{o}{=}\PY{l+m+mi}{1}\PY{p}{,}
                       \PY{n}{intercepts}\PY{o}{=}\PY{n}{intercepts}\PY{p}{,}
                       \PY{n}{max\PYZus{}rates}\PY{o}{=}\PY{n}{Uniform}\PY{p}{(}\PY{l+m+mi}{80}\PY{p}{,} \PY{l+m+mi}{100}\PY{p}{)}\PY{p}{,}
                       \PY{n}{encoders}\PY{o}{=}\PY{n}{encoders}\PY{p}{)}
    \PY{n}{nengo}\PY{o}{.}\PY{n}{Connection}\PY{p}{(}\PY{n+nb}{input}\PY{p}{,} \PY{n}{A}\PY{p}{)}
    \PY{n}{A\PYZus{}spikes} \PY{o}{=} \PY{n}{nengo}\PY{o}{.}\PY{n}{Probe}\PY{p}{(}\PY{n}{A}\PY{o}{.}\PY{n}{neurons}\PY{p}{,} \PY{n}{synapse}\PY{o}{=}\PY{l+m+mf}{0.01}\PY{p}{)}
\end{Verbatim}
\end{tcolorbox}

    \begin{tcolorbox}[breakable, size=fbox, boxrule=1pt, pad at break*=1mm,colback=cellbackground, colframe=cellborder]
\prompt{In}{incolor}{9}{\hspace{4pt}}
\begin{Verbatim}[commandchars=\\\{\}]
\PY{k}{with} \PY{n}{nengo}\PY{o}{.}\PY{n}{Simulator}\PY{p}{(}\PY{n}{model}\PY{p}{)} \PY{k}{as} \PY{n}{sim}\PY{p}{:}
    \PY{n}{sim}\PY{o}{.}\PY{n}{run}\PY{p}{(}\PY{l+m+mi}{1}\PY{p}{)}

\PY{n}{scale} \PY{o}{=} \PY{l+m+mi}{180}
\PY{n}{plt}\PY{o}{.}\PY{n}{figure}\PY{p}{(}\PY{p}{)}
\PY{k}{for} \PY{n}{i} \PY{o+ow}{in} \PY{n+nb}{range}\PY{p}{(}\PY{n}{A}\PY{o}{.}\PY{n}{n\PYZus{}neurons}\PY{p}{)}\PY{p}{:}
    \PY{n}{plt}\PY{o}{.}\PY{n}{plot}\PY{p}{(}\PY{n}{sim}\PY{o}{.}\PY{n}{trange}\PY{p}{(}\PY{p}{)}\PY{p}{,} \PY{n}{sim}\PY{o}{.}\PY{n}{data}\PY{p}{[}\PY{n}{A\PYZus{}spikes}\PY{p}{]}\PY{p}{[}\PY{p}{:}\PY{p}{,} \PY{n}{i}\PY{p}{]} \PY{o}{\PYZhy{}} \PY{n}{i} \PY{o}{*} \PY{n}{scale}\PY{p}{)}
\PY{n}{plt}\PY{o}{.}\PY{n}{xlim}\PY{p}{(}\PY{l+m+mi}{0}\PY{p}{,} \PY{l+m+mi}{1}\PY{p}{)}
\PY{n}{plt}\PY{o}{.}\PY{n}{ylim}\PY{p}{(}\PY{n}{scale} \PY{o}{*} \PY{p}{(}\PY{o}{\PYZhy{}}\PY{n}{A}\PY{o}{.}\PY{n}{n\PYZus{}neurons} \PY{o}{+} \PY{l+m+mi}{1}\PY{p}{)}\PY{p}{,} \PY{n}{scale}\PY{p}{)}
\PY{n}{plt}\PY{o}{.}\PY{n}{ylabel}\PY{p}{(}\PY{l+s+s2}{\PYZdq{}}\PY{l+s+s2}{Neuron}\PY{l+s+s2}{\PYZdq{}}\PY{p}{)}
\PY{n}{plt}\PY{o}{.}\PY{n}{yticks}\PY{p}{(}
    \PY{n}{np}\PY{o}{.}\PY{n}{arange}\PY{p}{(}\PY{n}{scale} \PY{o}{/} \PY{l+m+mf}{1.8}\PY{p}{,} \PY{p}{(}\PY{o}{\PYZhy{}}\PY{n}{A}\PY{o}{.}\PY{n}{n\PYZus{}neurons} \PY{o}{+} \PY{l+m+mi}{1}\PY{p}{)} \PY{o}{*} \PY{n}{scale}\PY{p}{,} \PY{o}{\PYZhy{}}\PY{n}{scale}\PY{p}{)}\PY{p}{,}
    \PY{n}{np}\PY{o}{.}\PY{n}{arange}\PY{p}{(}\PY{n}{A}\PY{o}{.}\PY{n}{n\PYZus{}neurons}\PY{p}{)}\PY{p}{)}
\PY{n}{hide\PYZus{}input}\PY{p}{(}\PY{p}{)}
\end{Verbatim}
\end{tcolorbox}

    
    \begin{verbatim}
HtmlProgressBar cannot be displayed. Please use the TerminalProgressBar. It can be enabled with `nengo.rc.set('progress', 'progress_bar', 'nengo.utils.progress.TerminalProgressBar')`.
    \end{verbatim}

    
    
    
    
    \begin{verbatim}
HtmlProgressBar cannot be displayed. Please use the TerminalProgressBar. It can be enabled with `nengo.rc.set('progress', 'progress_bar', 'nengo.utils.progress.TerminalProgressBar')`.
    \end{verbatim}

    
    
    
            \begin{tcolorbox}[breakable, boxrule=.5pt, size=fbox, pad at break*=1mm, opacityfill=0]
\prompt{Out}{outcolor}{9}{\hspace{3.5pt}}
\begin{Verbatim}[commandchars=\\\{\}]
<IPython.core.display.HTML object>
\end{Verbatim}
\end{tcolorbox}
        
    \begin{center}
    \adjustimage{max size={0.9\linewidth}{0.9\paperheight}}{intro_files/intro_14_5.png}
    \end{center}
    { \hspace*{\fill} \\}
    
    Then we mulitply those filtered spike trains with decoding weights and
sum them together to give an estimate of the input based on the spikes.

The decoding weights are determined by minimizing the squared difference
between the decoded estimate and the actual input signal.

    \begin{tcolorbox}[breakable, size=fbox, boxrule=1pt, pad at break*=1mm,colback=cellbackground, colframe=cellborder]
\prompt{In}{incolor}{10}{\hspace{4pt}}
\begin{Verbatim}[commandchars=\\\{\}]
\PY{k}{with} \PY{n}{model}\PY{p}{:}
    \PY{n}{A\PYZus{}probe} \PY{o}{=} \PY{n}{nengo}\PY{o}{.}\PY{n}{Probe}\PY{p}{(}\PY{n}{A}\PY{p}{,} \PY{n}{synapse}\PY{o}{=}\PY{l+m+mf}{0.01}\PY{p}{)}  \PY{c+c1}{\PYZsh{} 10ms PSC filter}
\end{Verbatim}
\end{tcolorbox}

    \begin{tcolorbox}[breakable, size=fbox, boxrule=1pt, pad at break*=1mm,colback=cellbackground, colframe=cellborder]
\prompt{In}{incolor}{11}{\hspace{4pt}}
\begin{Verbatim}[commandchars=\\\{\}]
\PY{k}{with} \PY{n}{nengo}\PY{o}{.}\PY{n}{Simulator}\PY{p}{(}\PY{n}{model}\PY{p}{)} \PY{k}{as} \PY{n}{sim}\PY{p}{:}
    \PY{n}{sim}\PY{o}{.}\PY{n}{run}\PY{p}{(}\PY{l+m+mi}{1}\PY{p}{)}

\PY{n}{plt}\PY{o}{.}\PY{n}{figure}\PY{p}{(}\PY{p}{)}
\PY{n}{plt}\PY{o}{.}\PY{n}{plot}\PY{p}{(}\PY{n}{sim}\PY{o}{.}\PY{n}{trange}\PY{p}{(}\PY{p}{)}\PY{p}{,} \PY{n}{sim}\PY{o}{.}\PY{n}{data}\PY{p}{[}\PY{n}{input\PYZus{}probe}\PY{p}{]}\PY{p}{,} \PY{n}{label}\PY{o}{=}\PY{l+s+s2}{\PYZdq{}}\PY{l+s+s2}{Input signal}\PY{l+s+s2}{\PYZdq{}}\PY{p}{)}
\PY{n}{plt}\PY{o}{.}\PY{n}{plot}\PY{p}{(}\PY{n}{sim}\PY{o}{.}\PY{n}{trange}\PY{p}{(}\PY{p}{)}\PY{p}{,} \PY{n}{sim}\PY{o}{.}\PY{n}{data}\PY{p}{[}\PY{n}{A\PYZus{}probe}\PY{p}{]}\PY{p}{,} \PY{n}{label}\PY{o}{=}\PY{l+s+s2}{\PYZdq{}}\PY{l+s+s2}{Decoded estimate}\PY{l+s+s2}{\PYZdq{}}\PY{p}{)}
\PY{n}{plt}\PY{o}{.}\PY{n}{legend}\PY{p}{(}\PY{n}{loc}\PY{o}{=}\PY{l+s+s2}{\PYZdq{}}\PY{l+s+s2}{best}\PY{l+s+s2}{\PYZdq{}}\PY{p}{)}
\PY{n}{plt}\PY{o}{.}\PY{n}{xlim}\PY{p}{(}\PY{l+m+mi}{0}\PY{p}{,} \PY{l+m+mi}{1}\PY{p}{)}
\PY{n}{hide\PYZus{}input}\PY{p}{(}\PY{p}{)}
\end{Verbatim}
\end{tcolorbox}

    
    \begin{verbatim}
HtmlProgressBar cannot be displayed. Please use the TerminalProgressBar. It can be enabled with `nengo.rc.set('progress', 'progress_bar', 'nengo.utils.progress.TerminalProgressBar')`.
    \end{verbatim}

    
    
    
    
    \begin{verbatim}
HtmlProgressBar cannot be displayed. Please use the TerminalProgressBar. It can be enabled with `nengo.rc.set('progress', 'progress_bar', 'nengo.utils.progress.TerminalProgressBar')`.
    \end{verbatim}

    
    
    
            \begin{tcolorbox}[breakable, boxrule=.5pt, size=fbox, pad at break*=1mm, opacityfill=0]
\prompt{Out}{outcolor}{11}{\hspace{3.5pt}}
\begin{Verbatim}[commandchars=\\\{\}]
<IPython.core.display.HTML object>
\end{Verbatim}
\end{tcolorbox}
        
    \begin{center}
    \adjustimage{max size={0.9\linewidth}{0.9\paperheight}}{intro_files/intro_17_5.png}
    \end{center}
    { \hspace*{\fill} \\}
    
    The accuracy of the decoded estimate increases as the number of neurons
increases.

    \begin{tcolorbox}[breakable, size=fbox, boxrule=1pt, pad at break*=1mm,colback=cellbackground, colframe=cellborder]
\prompt{In}{incolor}{12}{\hspace{4pt}}
\begin{Verbatim}[commandchars=\\\{\}]
\PY{n}{model} \PY{o}{=} \PY{n}{nengo}\PY{o}{.}\PY{n}{Network}\PY{p}{(}\PY{n}{label}\PY{o}{=}\PY{l+s+s2}{\PYZdq{}}\PY{l+s+s2}{NEF summary}\PY{l+s+s2}{\PYZdq{}}\PY{p}{)}
\PY{k}{with} \PY{n}{model}\PY{p}{:}
    \PY{n+nb}{input} \PY{o}{=} \PY{n}{nengo}\PY{o}{.}\PY{n}{Node}\PY{p}{(}\PY{k}{lambda} \PY{n}{t}\PY{p}{:} \PY{n}{t} \PY{o}{*} \PY{l+m+mi}{2} \PY{o}{\PYZhy{}} \PY{l+m+mi}{1}\PY{p}{)}
    \PY{n}{input\PYZus{}probe} \PY{o}{=} \PY{n}{nengo}\PY{o}{.}\PY{n}{Probe}\PY{p}{(}\PY{n+nb}{input}\PY{p}{)}
    \PY{n}{A} \PY{o}{=} \PY{n}{nengo}\PY{o}{.}\PY{n}{Ensemble}\PY{p}{(}\PY{l+m+mi}{30}\PY{p}{,} \PY{n}{dimensions}\PY{o}{=}\PY{l+m+mi}{1}\PY{p}{,} \PY{n}{max\PYZus{}rates}\PY{o}{=}\PY{n}{Uniform}\PY{p}{(}\PY{l+m+mi}{80}\PY{p}{,} \PY{l+m+mi}{100}\PY{p}{)}\PY{p}{)}
    \PY{n}{nengo}\PY{o}{.}\PY{n}{Connection}\PY{p}{(}\PY{n+nb}{input}\PY{p}{,} \PY{n}{A}\PY{p}{)}
    \PY{n}{A\PYZus{}spikes} \PY{o}{=} \PY{n}{nengo}\PY{o}{.}\PY{n}{Probe}\PY{p}{(}\PY{n}{A}\PY{o}{.}\PY{n}{neurons}\PY{p}{)}
    \PY{n}{A\PYZus{}probe} \PY{o}{=} \PY{n}{nengo}\PY{o}{.}\PY{n}{Probe}\PY{p}{(}\PY{n}{A}\PY{p}{,} \PY{n}{synapse}\PY{o}{=}\PY{l+m+mf}{0.01}\PY{p}{)}
\end{Verbatim}
\end{tcolorbox}

    \begin{tcolorbox}[breakable, size=fbox, boxrule=1pt, pad at break*=1mm,colback=cellbackground, colframe=cellborder]
\prompt{In}{incolor}{13}{\hspace{4pt}}
\begin{Verbatim}[commandchars=\\\{\}]
\PY{k}{with} \PY{n}{nengo}\PY{o}{.}\PY{n}{Simulator}\PY{p}{(}\PY{n}{model}\PY{p}{)} \PY{k}{as} \PY{n}{sim}\PY{p}{:}
    \PY{n}{sim}\PY{o}{.}\PY{n}{run}\PY{p}{(}\PY{l+m+mi}{1}\PY{p}{)}

\PY{n}{plt}\PY{o}{.}\PY{n}{figure}\PY{p}{(}\PY{n}{figsize}\PY{o}{=}\PY{p}{(}\PY{l+m+mi}{15}\PY{p}{,} \PY{l+m+mf}{3.5}\PY{p}{)}\PY{p}{)}

\PY{n}{plt}\PY{o}{.}\PY{n}{subplot}\PY{p}{(}\PY{l+m+mi}{1}\PY{p}{,} \PY{l+m+mi}{3}\PY{p}{,} \PY{l+m+mi}{1}\PY{p}{)}
\PY{n}{eval\PYZus{}points}\PY{p}{,} \PY{n}{activities} \PY{o}{=} \PY{n}{tuning\PYZus{}curves}\PY{p}{(}\PY{n}{A}\PY{p}{,} \PY{n}{sim}\PY{p}{)}
\PY{n}{plt}\PY{o}{.}\PY{n}{plot}\PY{p}{(}\PY{n}{eval\PYZus{}points}\PY{p}{,} \PY{n}{activities}\PY{p}{,} \PY{n}{lw}\PY{o}{=}\PY{l+m+mi}{2}\PY{p}{)}
\PY{n}{plt}\PY{o}{.}\PY{n}{xlabel}\PY{p}{(}\PY{l+s+s2}{\PYZdq{}}\PY{l+s+s2}{Input signal}\PY{l+s+s2}{\PYZdq{}}\PY{p}{)}
\PY{n}{plt}\PY{o}{.}\PY{n}{ylabel}\PY{p}{(}\PY{l+s+s2}{\PYZdq{}}\PY{l+s+s2}{Firing rate (Hz)}\PY{l+s+s2}{\PYZdq{}}\PY{p}{)}

\PY{n}{ax} \PY{o}{=} \PY{n}{plt}\PY{o}{.}\PY{n}{subplot}\PY{p}{(}\PY{l+m+mi}{1}\PY{p}{,} \PY{l+m+mi}{3}\PY{p}{,} \PY{l+m+mi}{2}\PY{p}{)}
\PY{n}{rasterplot}\PY{p}{(}\PY{n}{sim}\PY{o}{.}\PY{n}{trange}\PY{p}{(}\PY{p}{)}\PY{p}{,} \PY{n}{sim}\PY{o}{.}\PY{n}{data}\PY{p}{[}\PY{n}{A\PYZus{}spikes}\PY{p}{]}\PY{p}{,} \PY{n}{ax}\PY{p}{)}
\PY{n}{plt}\PY{o}{.}\PY{n}{xlim}\PY{p}{(}\PY{l+m+mi}{0}\PY{p}{,} \PY{l+m+mi}{1}\PY{p}{)}
\PY{n}{plt}\PY{o}{.}\PY{n}{xlabel}\PY{p}{(}\PY{l+s+s2}{\PYZdq{}}\PY{l+s+s2}{Time (s)}\PY{l+s+s2}{\PYZdq{}}\PY{p}{)}
\PY{n}{plt}\PY{o}{.}\PY{n}{ylabel}\PY{p}{(}\PY{l+s+s2}{\PYZdq{}}\PY{l+s+s2}{Neuron}\PY{l+s+s2}{\PYZdq{}}\PY{p}{)}

\PY{n}{plt}\PY{o}{.}\PY{n}{subplot}\PY{p}{(}\PY{l+m+mi}{1}\PY{p}{,} \PY{l+m+mi}{3}\PY{p}{,} \PY{l+m+mi}{3}\PY{p}{)}
\PY{n}{plt}\PY{o}{.}\PY{n}{plot}\PY{p}{(}\PY{n}{sim}\PY{o}{.}\PY{n}{trange}\PY{p}{(}\PY{p}{)}\PY{p}{,} \PY{n}{sim}\PY{o}{.}\PY{n}{data}\PY{p}{[}\PY{n}{input\PYZus{}probe}\PY{p}{]}\PY{p}{,} \PY{n}{label}\PY{o}{=}\PY{l+s+s2}{\PYZdq{}}\PY{l+s+s2}{Input signal}\PY{l+s+s2}{\PYZdq{}}\PY{p}{)}
\PY{n}{plt}\PY{o}{.}\PY{n}{plot}\PY{p}{(}\PY{n}{sim}\PY{o}{.}\PY{n}{trange}\PY{p}{(}\PY{p}{)}\PY{p}{,} \PY{n}{sim}\PY{o}{.}\PY{n}{data}\PY{p}{[}\PY{n}{A\PYZus{}probe}\PY{p}{]}\PY{p}{,} \PY{n}{label}\PY{o}{=}\PY{l+s+s2}{\PYZdq{}}\PY{l+s+s2}{Decoded esimate}\PY{l+s+s2}{\PYZdq{}}\PY{p}{)}
\PY{n}{plt}\PY{o}{.}\PY{n}{legend}\PY{p}{(}\PY{n}{loc}\PY{o}{=}\PY{l+s+s2}{\PYZdq{}}\PY{l+s+s2}{best}\PY{l+s+s2}{\PYZdq{}}\PY{p}{)}
\PY{n}{plt}\PY{o}{.}\PY{n}{xlabel}\PY{p}{(}\PY{l+s+s2}{\PYZdq{}}\PY{l+s+s2}{Time (s)}\PY{l+s+s2}{\PYZdq{}}\PY{p}{)}
\PY{n}{plt}\PY{o}{.}\PY{n}{xlim}\PY{p}{(}\PY{l+m+mi}{0}\PY{p}{,} \PY{l+m+mi}{1}\PY{p}{)}
\PY{n}{hide\PYZus{}input}\PY{p}{(}\PY{p}{)}
\end{Verbatim}
\end{tcolorbox}

    
    \begin{verbatim}
HtmlProgressBar cannot be displayed. Please use the TerminalProgressBar. It can be enabled with `nengo.rc.set('progress', 'progress_bar', 'nengo.utils.progress.TerminalProgressBar')`.
    \end{verbatim}

    
    
    
    
    \begin{verbatim}
HtmlProgressBar cannot be displayed. Please use the TerminalProgressBar. It can be enabled with `nengo.rc.set('progress', 'progress_bar', 'nengo.utils.progress.TerminalProgressBar')`.
    \end{verbatim}

    
    
    
            \begin{tcolorbox}[breakable, boxrule=.5pt, size=fbox, pad at break*=1mm, opacityfill=0]
\prompt{Out}{outcolor}{13}{\hspace{3.5pt}}
\begin{Verbatim}[commandchars=\\\{\}]
<IPython.core.display.HTML object>
\end{Verbatim}
\end{tcolorbox}
        
    \begin{center}
    \adjustimage{max size={0.9\linewidth}{0.9\paperheight}}{intro_files/intro_20_5.png}
    \end{center}
    { \hspace*{\fill} \\}
    
    Any smooth signal can be encoded and decoded.

    \begin{tcolorbox}[breakable, size=fbox, boxrule=1pt, pad at break*=1mm,colback=cellbackground, colframe=cellborder]
\prompt{In}{incolor}{14}{\hspace{4pt}}
\begin{Verbatim}[commandchars=\\\{\}]
\PY{n}{model} \PY{o}{=} \PY{n}{nengo}\PY{o}{.}\PY{n}{Network}\PY{p}{(}\PY{n}{label}\PY{o}{=}\PY{l+s+s2}{\PYZdq{}}\PY{l+s+s2}{NEF summary}\PY{l+s+s2}{\PYZdq{}}\PY{p}{)}
\PY{k}{with} \PY{n}{model}\PY{p}{:}
    \PY{n+nb}{input} \PY{o}{=} \PY{n}{nengo}\PY{o}{.}\PY{n}{Node}\PY{p}{(}\PY{n}{WhiteSignal}\PY{p}{(}\PY{l+m+mi}{1}\PY{p}{,} \PY{n}{high}\PY{o}{=}\PY{l+m+mi}{5}\PY{p}{)}\PY{p}{,} \PY{n}{size\PYZus{}out}\PY{o}{=}\PY{l+m+mi}{1}\PY{p}{)}
    \PY{n}{input\PYZus{}probe} \PY{o}{=} \PY{n}{nengo}\PY{o}{.}\PY{n}{Probe}\PY{p}{(}\PY{n+nb}{input}\PY{p}{)}
    \PY{n}{A} \PY{o}{=} \PY{n}{nengo}\PY{o}{.}\PY{n}{Ensemble}\PY{p}{(}\PY{l+m+mi}{30}\PY{p}{,} \PY{n}{dimensions}\PY{o}{=}\PY{l+m+mi}{1}\PY{p}{,} \PY{n}{max\PYZus{}rates}\PY{o}{=}\PY{n}{Uniform}\PY{p}{(}\PY{l+m+mi}{80}\PY{p}{,} \PY{l+m+mi}{100}\PY{p}{)}\PY{p}{)}
    \PY{n}{nengo}\PY{o}{.}\PY{n}{Connection}\PY{p}{(}\PY{n+nb}{input}\PY{p}{,} \PY{n}{A}\PY{p}{)}
    \PY{n}{A\PYZus{}spikes} \PY{o}{=} \PY{n}{nengo}\PY{o}{.}\PY{n}{Probe}\PY{p}{(}\PY{n}{A}\PY{o}{.}\PY{n}{neurons}\PY{p}{)}
    \PY{n}{A\PYZus{}probe} \PY{o}{=} \PY{n}{nengo}\PY{o}{.}\PY{n}{Probe}\PY{p}{(}\PY{n}{A}\PY{p}{,} \PY{n}{synapse}\PY{o}{=}\PY{l+m+mf}{0.01}\PY{p}{)}
\end{Verbatim}
\end{tcolorbox}

    \begin{tcolorbox}[breakable, size=fbox, boxrule=1pt, pad at break*=1mm,colback=cellbackground, colframe=cellborder]
\prompt{In}{incolor}{15}{\hspace{4pt}}
\begin{Verbatim}[commandchars=\\\{\}]
\PY{k}{with} \PY{n}{nengo}\PY{o}{.}\PY{n}{Simulator}\PY{p}{(}\PY{n}{model}\PY{p}{)} \PY{k}{as} \PY{n}{sim}\PY{p}{:}
    \PY{n}{sim}\PY{o}{.}\PY{n}{run}\PY{p}{(}\PY{l+m+mi}{1}\PY{p}{)}

\PY{n}{plt}\PY{o}{.}\PY{n}{figure}\PY{p}{(}\PY{n}{figsize}\PY{o}{=}\PY{p}{(}\PY{l+m+mi}{10}\PY{p}{,} \PY{l+m+mf}{3.5}\PY{p}{)}\PY{p}{)}
\PY{n}{plt}\PY{o}{.}\PY{n}{subplot}\PY{p}{(}\PY{l+m+mi}{1}\PY{p}{,} \PY{l+m+mi}{2}\PY{p}{,} \PY{l+m+mi}{1}\PY{p}{)}
\PY{n}{plt}\PY{o}{.}\PY{n}{plot}\PY{p}{(}\PY{n}{sim}\PY{o}{.}\PY{n}{trange}\PY{p}{(}\PY{p}{)}\PY{p}{,} \PY{n}{sim}\PY{o}{.}\PY{n}{data}\PY{p}{[}\PY{n}{input\PYZus{}probe}\PY{p}{]}\PY{p}{,} \PY{n}{label}\PY{o}{=}\PY{l+s+s2}{\PYZdq{}}\PY{l+s+s2}{Input signal}\PY{l+s+s2}{\PYZdq{}}\PY{p}{)}
\PY{n}{plt}\PY{o}{.}\PY{n}{plot}\PY{p}{(}\PY{n}{sim}\PY{o}{.}\PY{n}{trange}\PY{p}{(}\PY{p}{)}\PY{p}{,} \PY{n}{sim}\PY{o}{.}\PY{n}{data}\PY{p}{[}\PY{n}{A\PYZus{}probe}\PY{p}{]}\PY{p}{,} \PY{n}{label}\PY{o}{=}\PY{l+s+s2}{\PYZdq{}}\PY{l+s+s2}{Decoded esimate}\PY{l+s+s2}{\PYZdq{}}\PY{p}{)}
\PY{n}{plt}\PY{o}{.}\PY{n}{legend}\PY{p}{(}\PY{n}{loc}\PY{o}{=}\PY{l+s+s2}{\PYZdq{}}\PY{l+s+s2}{best}\PY{l+s+s2}{\PYZdq{}}\PY{p}{)}
\PY{n}{plt}\PY{o}{.}\PY{n}{xlabel}\PY{p}{(}\PY{l+s+s2}{\PYZdq{}}\PY{l+s+s2}{Time (s)}\PY{l+s+s2}{\PYZdq{}}\PY{p}{)}
\PY{n}{plt}\PY{o}{.}\PY{n}{xlim}\PY{p}{(}\PY{l+m+mi}{0}\PY{p}{,} \PY{l+m+mi}{1}\PY{p}{)}

\PY{n}{ax} \PY{o}{=} \PY{n}{plt}\PY{o}{.}\PY{n}{subplot}\PY{p}{(}\PY{l+m+mi}{1}\PY{p}{,} \PY{l+m+mi}{2}\PY{p}{,} \PY{l+m+mi}{2}\PY{p}{)}
\PY{n}{rasterplot}\PY{p}{(}\PY{n}{sim}\PY{o}{.}\PY{n}{trange}\PY{p}{(}\PY{p}{)}\PY{p}{,} \PY{n}{sim}\PY{o}{.}\PY{n}{data}\PY{p}{[}\PY{n}{A\PYZus{}spikes}\PY{p}{]}\PY{p}{,} \PY{n}{ax}\PY{p}{)}
\PY{n}{plt}\PY{o}{.}\PY{n}{xlim}\PY{p}{(}\PY{l+m+mi}{0}\PY{p}{,} \PY{l+m+mi}{1}\PY{p}{)}
\PY{n}{plt}\PY{o}{.}\PY{n}{xlabel}\PY{p}{(}\PY{l+s+s2}{\PYZdq{}}\PY{l+s+s2}{Time (s)}\PY{l+s+s2}{\PYZdq{}}\PY{p}{)}
\PY{n}{plt}\PY{o}{.}\PY{n}{ylabel}\PY{p}{(}\PY{l+s+s2}{\PYZdq{}}\PY{l+s+s2}{Neuron}\PY{l+s+s2}{\PYZdq{}}\PY{p}{)}
\PY{n}{hide\PYZus{}input}\PY{p}{(}\PY{p}{)}
\end{Verbatim}
\end{tcolorbox}

    
    \begin{verbatim}
HtmlProgressBar cannot be displayed. Please use the TerminalProgressBar. It can be enabled with `nengo.rc.set('progress', 'progress_bar', 'nengo.utils.progress.TerminalProgressBar')`.
    \end{verbatim}

    
    
    
    
    \begin{verbatim}
HtmlProgressBar cannot be displayed. Please use the TerminalProgressBar. It can be enabled with `nengo.rc.set('progress', 'progress_bar', 'nengo.utils.progress.TerminalProgressBar')`.
    \end{verbatim}

    
    
    
            \begin{tcolorbox}[breakable, boxrule=.5pt, size=fbox, pad at break*=1mm, opacityfill=0]
\prompt{Out}{outcolor}{15}{\hspace{3.5pt}}
\begin{Verbatim}[commandchars=\\\{\}]
<IPython.core.display.HTML object>
\end{Verbatim}
\end{tcolorbox}
        
    \begin{center}
    \adjustimage{max size={0.9\linewidth}{0.9\paperheight}}{intro_files/intro_23_5.png}
    \end{center}
    { \hspace*{\fill} \\}
    
    \hypertarget{principle-2-transformation}{%
\subsection{Principle 2:
Transformation}\label{principle-2-transformation}}

Encoding and decoding allow us to encode signals over time, and decode
transformations of those signals.

In fact, we can decode arbitrary transformations of the input signal,
not just the signal itself (as in the previous example).

Let's decode the square of our white noise input.

    \begin{tcolorbox}[breakable, size=fbox, boxrule=1pt, pad at break*=1mm,colback=cellbackground, colframe=cellborder]
\prompt{In}{incolor}{16}{\hspace{4pt}}
\begin{Verbatim}[commandchars=\\\{\}]
\PY{n}{model} \PY{o}{=} \PY{n}{nengo}\PY{o}{.}\PY{n}{Network}\PY{p}{(}\PY{n}{label}\PY{o}{=}\PY{l+s+s2}{\PYZdq{}}\PY{l+s+s2}{NEF summary}\PY{l+s+s2}{\PYZdq{}}\PY{p}{)}
\PY{k}{with} \PY{n}{model}\PY{p}{:}
    \PY{n+nb}{input} \PY{o}{=} \PY{n}{nengo}\PY{o}{.}\PY{n}{Node}\PY{p}{(}\PY{n}{WhiteSignal}\PY{p}{(}\PY{l+m+mi}{1}\PY{p}{,} \PY{n}{high}\PY{o}{=}\PY{l+m+mi}{5}\PY{p}{)}\PY{p}{,} \PY{n}{size\PYZus{}out}\PY{o}{=}\PY{l+m+mi}{1}\PY{p}{)}
    \PY{n}{input\PYZus{}probe} \PY{o}{=} \PY{n}{nengo}\PY{o}{.}\PY{n}{Probe}\PY{p}{(}\PY{n+nb}{input}\PY{p}{,} \PY{p}{)}
    \PY{n}{A} \PY{o}{=} \PY{n}{nengo}\PY{o}{.}\PY{n}{Ensemble}\PY{p}{(}\PY{l+m+mi}{30}\PY{p}{,} \PY{n}{dimensions}\PY{o}{=}\PY{l+m+mi}{1}\PY{p}{,} \PY{n}{max\PYZus{}rates}\PY{o}{=}\PY{n}{Uniform}\PY{p}{(}\PY{l+m+mi}{80}\PY{p}{,} \PY{l+m+mi}{100}\PY{p}{)}\PY{p}{)}
    \PY{n}{Asquare} \PY{o}{=} \PY{n}{nengo}\PY{o}{.}\PY{n}{Node}\PY{p}{(}\PY{n}{size\PYZus{}in}\PY{o}{=}\PY{l+m+mi}{1}\PY{p}{)}
    \PY{n}{nengo}\PY{o}{.}\PY{n}{Connection}\PY{p}{(}\PY{n+nb}{input}\PY{p}{,} \PY{n}{A}\PY{p}{)}
    \PY{n}{nengo}\PY{o}{.}\PY{n}{Connection}\PY{p}{(}\PY{n}{A}\PY{p}{,} \PY{n}{Asquare}\PY{p}{,} \PY{n}{function}\PY{o}{=}\PY{n}{np}\PY{o}{.}\PY{n}{square}\PY{p}{)}
    \PY{n}{A\PYZus{}spikes} \PY{o}{=} \PY{n}{nengo}\PY{o}{.}\PY{n}{Probe}\PY{p}{(}\PY{n}{A}\PY{o}{.}\PY{n}{neurons}\PY{p}{)}
    \PY{n}{Asquare\PYZus{}probe} \PY{o}{=} \PY{n}{nengo}\PY{o}{.}\PY{n}{Probe}\PY{p}{(}\PY{n}{Asquare}\PY{p}{,} \PY{n}{synapse}\PY{o}{=}\PY{l+m+mf}{0.01}\PY{p}{)}
\end{Verbatim}
\end{tcolorbox}

    \begin{tcolorbox}[breakable, size=fbox, boxrule=1pt, pad at break*=1mm,colback=cellbackground, colframe=cellborder]
\prompt{In}{incolor}{17}{\hspace{4pt}}
\begin{Verbatim}[commandchars=\\\{\}]
\PY{k}{with} \PY{n}{nengo}\PY{o}{.}\PY{n}{Simulator}\PY{p}{(}\PY{n}{model}\PY{p}{)} \PY{k}{as} \PY{n}{sim}\PY{p}{:}
    \PY{n}{sim}\PY{o}{.}\PY{n}{run}\PY{p}{(}\PY{l+m+mi}{1}\PY{p}{)}

\PY{n}{plt}\PY{o}{.}\PY{n}{figure}\PY{p}{(}\PY{n}{figsize}\PY{o}{=}\PY{p}{(}\PY{l+m+mi}{10}\PY{p}{,} \PY{l+m+mf}{3.5}\PY{p}{)}\PY{p}{)}
\PY{n}{plt}\PY{o}{.}\PY{n}{subplot}\PY{p}{(}\PY{l+m+mi}{1}\PY{p}{,} \PY{l+m+mi}{2}\PY{p}{,} \PY{l+m+mi}{1}\PY{p}{)}
\PY{n}{plt}\PY{o}{.}\PY{n}{plot}\PY{p}{(}
    \PY{n}{sim}\PY{o}{.}\PY{n}{trange}\PY{p}{(}\PY{p}{)}\PY{p}{,}
    \PY{n}{sim}\PY{o}{.}\PY{n}{data}\PY{p}{[}\PY{n}{input\PYZus{}probe}\PY{p}{]}\PY{p}{,}
    \PY{n}{label}\PY{o}{=}\PY{l+s+s2}{\PYZdq{}}\PY{l+s+s2}{Input signal}\PY{l+s+s2}{\PYZdq{}}\PY{p}{)}
\PY{n}{plt}\PY{o}{.}\PY{n}{plot}\PY{p}{(}
    \PY{n}{sim}\PY{o}{.}\PY{n}{trange}\PY{p}{(}\PY{p}{)}\PY{p}{,}
    \PY{n}{sim}\PY{o}{.}\PY{n}{data}\PY{p}{[}\PY{n}{Asquare\PYZus{}probe}\PY{p}{]}\PY{p}{,}
    \PY{n}{label}\PY{o}{=}\PY{l+s+s2}{\PYZdq{}}\PY{l+s+s2}{Decoded esimate}\PY{l+s+s2}{\PYZdq{}}\PY{p}{)}
\PY{n}{plt}\PY{o}{.}\PY{n}{plot}\PY{p}{(}
    \PY{n}{sim}\PY{o}{.}\PY{n}{trange}\PY{p}{(}\PY{p}{)}\PY{p}{,}
    \PY{n}{np}\PY{o}{.}\PY{n}{square}\PY{p}{(}\PY{n}{sim}\PY{o}{.}\PY{n}{data}\PY{p}{[}\PY{n}{input\PYZus{}probe}\PY{p}{]}\PY{p}{)}\PY{p}{,}
    \PY{n}{label}\PY{o}{=}\PY{l+s+s2}{\PYZdq{}}\PY{l+s+s2}{Input signal squared}\PY{l+s+s2}{\PYZdq{}}\PY{p}{)}
\PY{n}{plt}\PY{o}{.}\PY{n}{legend}\PY{p}{(}\PY{n}{loc}\PY{o}{=}\PY{l+s+s2}{\PYZdq{}}\PY{l+s+s2}{best}\PY{l+s+s2}{\PYZdq{}}\PY{p}{,} \PY{n}{fontsize}\PY{o}{=}\PY{l+s+s1}{\PYZsq{}}\PY{l+s+s1}{medium}\PY{l+s+s1}{\PYZsq{}}\PY{p}{)}
\PY{n}{plt}\PY{o}{.}\PY{n}{xlabel}\PY{p}{(}\PY{l+s+s2}{\PYZdq{}}\PY{l+s+s2}{Time (s)}\PY{l+s+s2}{\PYZdq{}}\PY{p}{)}
\PY{n}{plt}\PY{o}{.}\PY{n}{xlim}\PY{p}{(}\PY{l+m+mi}{0}\PY{p}{,} \PY{l+m+mi}{1}\PY{p}{)}

\PY{n}{ax} \PY{o}{=} \PY{n}{plt}\PY{o}{.}\PY{n}{subplot}\PY{p}{(}\PY{l+m+mi}{1}\PY{p}{,} \PY{l+m+mi}{2}\PY{p}{,} \PY{l+m+mi}{2}\PY{p}{)}
\PY{n}{rasterplot}\PY{p}{(}\PY{n}{sim}\PY{o}{.}\PY{n}{trange}\PY{p}{(}\PY{p}{)}\PY{p}{,} \PY{n}{sim}\PY{o}{.}\PY{n}{data}\PY{p}{[}\PY{n}{A\PYZus{}spikes}\PY{p}{]}\PY{p}{)}
\PY{n}{plt}\PY{o}{.}\PY{n}{xlim}\PY{p}{(}\PY{l+m+mi}{0}\PY{p}{,} \PY{l+m+mi}{1}\PY{p}{)}
\PY{n}{plt}\PY{o}{.}\PY{n}{xlabel}\PY{p}{(}\PY{l+s+s2}{\PYZdq{}}\PY{l+s+s2}{Time (s)}\PY{l+s+s2}{\PYZdq{}}\PY{p}{)}
\PY{n}{plt}\PY{o}{.}\PY{n}{ylabel}\PY{p}{(}\PY{l+s+s2}{\PYZdq{}}\PY{l+s+s2}{Neuron}\PY{l+s+s2}{\PYZdq{}}\PY{p}{)}
\PY{n}{hide\PYZus{}input}\PY{p}{(}\PY{p}{)}
\end{Verbatim}
\end{tcolorbox}

    
    \begin{verbatim}
HtmlProgressBar cannot be displayed. Please use the TerminalProgressBar. It can be enabled with `nengo.rc.set('progress', 'progress_bar', 'nengo.utils.progress.TerminalProgressBar')`.
    \end{verbatim}

    
    
    
    
    \begin{verbatim}
HtmlProgressBar cannot be displayed. Please use the TerminalProgressBar. It can be enabled with `nengo.rc.set('progress', 'progress_bar', 'nengo.utils.progress.TerminalProgressBar')`.
    \end{verbatim}

    
    
    
            \begin{tcolorbox}[breakable, boxrule=.5pt, size=fbox, pad at break*=1mm, opacityfill=0]
\prompt{Out}{outcolor}{17}{\hspace{3.5pt}}
\begin{Verbatim}[commandchars=\\\{\}]
<IPython.core.display.HTML object>
\end{Verbatim}
\end{tcolorbox}
        
    \begin{center}
    \adjustimage{max size={0.9\linewidth}{0.9\paperheight}}{intro_files/intro_26_5.png}
    \end{center}
    { \hspace*{\fill} \\}
    
    Notice that the spike trains are exactly the same. The only difference
is how we're interpreting those spikes. We told Nengo to compute a new
set of decoders that estimate the function \(x^2\).

In general, the transformation principle determines how we can decode
spike trains to compute linear and nonlinear transformations of signals
encoded in a population of neurons. We can then project those
transformed signals into another population, and repeat the process.
Essentially, this provides a means of computing the neural connection
weights to compute an arbitrary function between populations.

Suppose we are representing a sine wave.

    \begin{tcolorbox}[breakable, size=fbox, boxrule=1pt, pad at break*=1mm,colback=cellbackground, colframe=cellborder]
\prompt{In}{incolor}{18}{\hspace{4pt}}
\begin{Verbatim}[commandchars=\\\{\}]
\PY{n}{model} \PY{o}{=} \PY{n}{nengo}\PY{o}{.}\PY{n}{Network}\PY{p}{(}\PY{n}{label}\PY{o}{=}\PY{l+s+s2}{\PYZdq{}}\PY{l+s+s2}{NEF summary}\PY{l+s+s2}{\PYZdq{}}\PY{p}{)}
\PY{k}{with} \PY{n}{model}\PY{p}{:}
    \PY{n+nb}{input} \PY{o}{=} \PY{n}{nengo}\PY{o}{.}\PY{n}{Node}\PY{p}{(}\PY{k}{lambda} \PY{n}{t}\PY{p}{:} \PY{n}{np}\PY{o}{.}\PY{n}{sin}\PY{p}{(}\PY{n}{np}\PY{o}{.}\PY{n}{pi} \PY{o}{*} \PY{n}{t}\PY{p}{)}\PY{p}{)}
    \PY{n}{A} \PY{o}{=} \PY{n}{nengo}\PY{o}{.}\PY{n}{Ensemble}\PY{p}{(}\PY{l+m+mi}{30}\PY{p}{,} \PY{n}{dimensions}\PY{o}{=}\PY{l+m+mi}{1}\PY{p}{,} \PY{n}{max\PYZus{}rates}\PY{o}{=}\PY{n}{Uniform}\PY{p}{(}\PY{l+m+mi}{80}\PY{p}{,} \PY{l+m+mi}{100}\PY{p}{)}\PY{p}{)}
    \PY{n}{nengo}\PY{o}{.}\PY{n}{Connection}\PY{p}{(}\PY{n+nb}{input}\PY{p}{,} \PY{n}{A}\PY{p}{)}
    \PY{n}{A\PYZus{}spikes} \PY{o}{=} \PY{n}{nengo}\PY{o}{.}\PY{n}{Probe}\PY{p}{(}\PY{n}{A}\PY{o}{.}\PY{n}{neurons}\PY{p}{)}
    \PY{n}{A\PYZus{}probe} \PY{o}{=} \PY{n}{nengo}\PY{o}{.}\PY{n}{Probe}\PY{p}{(}\PY{n}{A}\PY{p}{,} \PY{n}{synapse}\PY{o}{=}\PY{l+m+mf}{0.01}\PY{p}{)}
\end{Verbatim}
\end{tcolorbox}

    \begin{tcolorbox}[breakable, size=fbox, boxrule=1pt, pad at break*=1mm,colback=cellbackground, colframe=cellborder]
\prompt{In}{incolor}{19}{\hspace{4pt}}
\begin{Verbatim}[commandchars=\\\{\}]
\PY{k}{with} \PY{n}{nengo}\PY{o}{.}\PY{n}{Simulator}\PY{p}{(}\PY{n}{model}\PY{p}{)} \PY{k}{as} \PY{n}{sim}\PY{p}{:}
    \PY{n}{sim}\PY{o}{.}\PY{n}{run}\PY{p}{(}\PY{l+m+mi}{2}\PY{p}{)}

\PY{n}{plt}\PY{o}{.}\PY{n}{figure}\PY{p}{(}\PY{n}{figsize}\PY{o}{=}\PY{p}{(}\PY{l+m+mi}{10}\PY{p}{,} \PY{l+m+mf}{3.5}\PY{p}{)}\PY{p}{)}
\PY{n}{plt}\PY{o}{.}\PY{n}{subplot}\PY{p}{(}\PY{l+m+mi}{1}\PY{p}{,} \PY{l+m+mi}{2}\PY{p}{,} \PY{l+m+mi}{1}\PY{p}{)}
\PY{n}{plt}\PY{o}{.}\PY{n}{plot}\PY{p}{(}\PY{n}{sim}\PY{o}{.}\PY{n}{trange}\PY{p}{(}\PY{p}{)}\PY{p}{,} \PY{n}{sim}\PY{o}{.}\PY{n}{data}\PY{p}{[}\PY{n}{A\PYZus{}probe}\PY{p}{]}\PY{p}{)}
\PY{n}{plt}\PY{o}{.}\PY{n}{title}\PY{p}{(}\PY{l+s+s2}{\PYZdq{}}\PY{l+s+s2}{A}\PY{l+s+s2}{\PYZdq{}}\PY{p}{)}
\PY{n}{plt}\PY{o}{.}\PY{n}{xlabel}\PY{p}{(}\PY{l+s+s2}{\PYZdq{}}\PY{l+s+s2}{Time (s)}\PY{l+s+s2}{\PYZdq{}}\PY{p}{)}
\PY{n}{plt}\PY{o}{.}\PY{n}{xlim}\PY{p}{(}\PY{l+m+mi}{0}\PY{p}{,} \PY{l+m+mi}{2}\PY{p}{)}

\PY{n}{ax} \PY{o}{=} \PY{n}{plt}\PY{o}{.}\PY{n}{subplot}\PY{p}{(}\PY{l+m+mi}{1}\PY{p}{,} \PY{l+m+mi}{2}\PY{p}{,} \PY{l+m+mi}{2}\PY{p}{)}
\PY{n}{rasterplot}\PY{p}{(}\PY{n}{sim}\PY{o}{.}\PY{n}{trange}\PY{p}{(}\PY{p}{)}\PY{p}{,} \PY{n}{sim}\PY{o}{.}\PY{n}{data}\PY{p}{[}\PY{n}{A\PYZus{}spikes}\PY{p}{]}\PY{p}{,} \PY{n}{ax}\PY{p}{)}
\PY{n}{plt}\PY{o}{.}\PY{n}{xlim}\PY{p}{(}\PY{l+m+mi}{0}\PY{p}{,} \PY{l+m+mi}{2}\PY{p}{)}
\PY{n}{plt}\PY{o}{.}\PY{n}{title}\PY{p}{(}\PY{l+s+s2}{\PYZdq{}}\PY{l+s+s2}{A}\PY{l+s+s2}{\PYZdq{}}\PY{p}{)}
\PY{n}{plt}\PY{o}{.}\PY{n}{xlabel}\PY{p}{(}\PY{l+s+s2}{\PYZdq{}}\PY{l+s+s2}{Time (s)}\PY{l+s+s2}{\PYZdq{}}\PY{p}{)}
\PY{n}{plt}\PY{o}{.}\PY{n}{ylabel}\PY{p}{(}\PY{l+s+s2}{\PYZdq{}}\PY{l+s+s2}{Neuron}\PY{l+s+s2}{\PYZdq{}}\PY{p}{)}
\PY{n}{hide\PYZus{}input}\PY{p}{(}\PY{p}{)}
\end{Verbatim}
\end{tcolorbox}

    
    \begin{verbatim}
HtmlProgressBar cannot be displayed. Please use the TerminalProgressBar. It can be enabled with `nengo.rc.set('progress', 'progress_bar', 'nengo.utils.progress.TerminalProgressBar')`.
    \end{verbatim}

    
    
    
    
    \begin{verbatim}
HtmlProgressBar cannot be displayed. Please use the TerminalProgressBar. It can be enabled with `nengo.rc.set('progress', 'progress_bar', 'nengo.utils.progress.TerminalProgressBar')`.
    \end{verbatim}

    
    
    
            \begin{tcolorbox}[breakable, boxrule=.5pt, size=fbox, pad at break*=1mm, opacityfill=0]
\prompt{Out}{outcolor}{19}{\hspace{3.5pt}}
\begin{Verbatim}[commandchars=\\\{\}]
<IPython.core.display.HTML object>
\end{Verbatim}
\end{tcolorbox}
        
    \begin{center}
    \adjustimage{max size={0.9\linewidth}{0.9\paperheight}}{intro_files/intro_29_5.png}
    \end{center}
    { \hspace*{\fill} \\}
    
    Linear transformations of that signal involve solving for the usual
decoders, and scaling those decoding weights. Let us flip this sine wave
upside down as it is transmitted between two populations
(i.e.~population A and population -A).

    \begin{tcolorbox}[breakable, size=fbox, boxrule=1pt, pad at break*=1mm,colback=cellbackground, colframe=cellborder]
\prompt{In}{incolor}{20}{\hspace{4pt}}
\begin{Verbatim}[commandchars=\\\{\}]
\PY{k}{with} \PY{n}{model}\PY{p}{:}
    \PY{n}{minusA} \PY{o}{=} \PY{n}{nengo}\PY{o}{.}\PY{n}{Ensemble}\PY{p}{(}\PY{l+m+mi}{30}\PY{p}{,} \PY{n}{dimensions}\PY{o}{=}\PY{l+m+mi}{1}\PY{p}{,} \PY{n}{max\PYZus{}rates}\PY{o}{=}\PY{n}{Uniform}\PY{p}{(}\PY{l+m+mi}{80}\PY{p}{,} \PY{l+m+mi}{100}\PY{p}{)}\PY{p}{)}
    \PY{n}{nengo}\PY{o}{.}\PY{n}{Connection}\PY{p}{(}\PY{n}{A}\PY{p}{,} \PY{n}{minusA}\PY{p}{,} \PY{n}{function}\PY{o}{=}\PY{k}{lambda} \PY{n}{x}\PY{p}{:} \PY{o}{\PYZhy{}}\PY{n}{x}\PY{p}{)}
    \PY{n}{minusA\PYZus{}spikes} \PY{o}{=} \PY{n}{nengo}\PY{o}{.}\PY{n}{Probe}\PY{p}{(}\PY{n}{minusA}\PY{o}{.}\PY{n}{neurons}\PY{p}{)}
    \PY{n}{minusA\PYZus{}probe} \PY{o}{=} \PY{n}{nengo}\PY{o}{.}\PY{n}{Probe}\PY{p}{(}\PY{n}{minusA}\PY{p}{,} \PY{n}{synapse}\PY{o}{=}\PY{l+m+mf}{0.01}\PY{p}{)}
\end{Verbatim}
\end{tcolorbox}

    \begin{tcolorbox}[breakable, size=fbox, boxrule=1pt, pad at break*=1mm,colback=cellbackground, colframe=cellborder]
\prompt{In}{incolor}{21}{\hspace{4pt}}
\begin{Verbatim}[commandchars=\\\{\}]
\PY{k}{with} \PY{n}{nengo}\PY{o}{.}\PY{n}{Simulator}\PY{p}{(}\PY{n}{model}\PY{p}{)} \PY{k}{as} \PY{n}{sim}\PY{p}{:}
    \PY{n}{sim}\PY{o}{.}\PY{n}{run}\PY{p}{(}\PY{l+m+mi}{2}\PY{p}{)}

\PY{n}{plt}\PY{o}{.}\PY{n}{figure}\PY{p}{(}\PY{n}{figsize}\PY{o}{=}\PY{p}{(}\PY{l+m+mi}{10}\PY{p}{,} \PY{l+m+mi}{5}\PY{p}{)}\PY{p}{)}
\PY{n}{plt}\PY{o}{.}\PY{n}{subplot}\PY{p}{(}\PY{l+m+mi}{2}\PY{p}{,} \PY{l+m+mi}{2}\PY{p}{,} \PY{l+m+mi}{1}\PY{p}{)}
\PY{n}{plt}\PY{o}{.}\PY{n}{plot}\PY{p}{(}\PY{n}{sim}\PY{o}{.}\PY{n}{trange}\PY{p}{(}\PY{p}{)}\PY{p}{,} \PY{n}{sim}\PY{o}{.}\PY{n}{data}\PY{p}{[}\PY{n}{A\PYZus{}probe}\PY{p}{]}\PY{p}{)}
\PY{n}{plt}\PY{o}{.}\PY{n}{title}\PY{p}{(}\PY{l+s+s2}{\PYZdq{}}\PY{l+s+s2}{A}\PY{l+s+s2}{\PYZdq{}}\PY{p}{)}
\PY{n}{plt}\PY{o}{.}\PY{n}{xticks}\PY{p}{(}\PY{p}{(}\PY{p}{)}\PY{p}{)}
\PY{n}{plt}\PY{o}{.}\PY{n}{xlim}\PY{p}{(}\PY{l+m+mi}{0}\PY{p}{,} \PY{l+m+mi}{2}\PY{p}{)}

\PY{n}{plt}\PY{o}{.}\PY{n}{subplot}\PY{p}{(}\PY{l+m+mi}{2}\PY{p}{,} \PY{l+m+mi}{2}\PY{p}{,} \PY{l+m+mi}{3}\PY{p}{)}
\PY{n}{plt}\PY{o}{.}\PY{n}{plot}\PY{p}{(}\PY{n}{sim}\PY{o}{.}\PY{n}{trange}\PY{p}{(}\PY{p}{)}\PY{p}{,} \PY{n}{sim}\PY{o}{.}\PY{n}{data}\PY{p}{[}\PY{n}{minusA\PYZus{}probe}\PY{p}{]}\PY{p}{)}
\PY{n}{plt}\PY{o}{.}\PY{n}{title}\PY{p}{(}\PY{l+s+s2}{\PYZdq{}}\PY{l+s+s2}{\PYZhy{}A}\PY{l+s+s2}{\PYZdq{}}\PY{p}{)}
\PY{n}{plt}\PY{o}{.}\PY{n}{xlabel}\PY{p}{(}\PY{l+s+s2}{\PYZdq{}}\PY{l+s+s2}{Time (s)}\PY{l+s+s2}{\PYZdq{}}\PY{p}{)}
\PY{n}{plt}\PY{o}{.}\PY{n}{xlim}\PY{p}{(}\PY{l+m+mi}{0}\PY{p}{,} \PY{l+m+mi}{2}\PY{p}{)}

\PY{n}{ax} \PY{o}{=} \PY{n}{plt}\PY{o}{.}\PY{n}{subplot}\PY{p}{(}\PY{l+m+mi}{2}\PY{p}{,} \PY{l+m+mi}{2}\PY{p}{,} \PY{l+m+mi}{2}\PY{p}{)}
\PY{n}{rasterplot}\PY{p}{(}\PY{n}{sim}\PY{o}{.}\PY{n}{trange}\PY{p}{(}\PY{p}{)}\PY{p}{,} \PY{n}{sim}\PY{o}{.}\PY{n}{data}\PY{p}{[}\PY{n}{A\PYZus{}spikes}\PY{p}{]}\PY{p}{,} \PY{n}{ax}\PY{p}{)}
\PY{n}{plt}\PY{o}{.}\PY{n}{xlim}\PY{p}{(}\PY{l+m+mi}{0}\PY{p}{,} \PY{l+m+mi}{2}\PY{p}{)}
\PY{n}{plt}\PY{o}{.}\PY{n}{title}\PY{p}{(}\PY{l+s+s2}{\PYZdq{}}\PY{l+s+s2}{A}\PY{l+s+s2}{\PYZdq{}}\PY{p}{)}
\PY{n}{plt}\PY{o}{.}\PY{n}{xticks}\PY{p}{(}\PY{p}{(}\PY{p}{)}\PY{p}{)}
\PY{n}{plt}\PY{o}{.}\PY{n}{ylabel}\PY{p}{(}\PY{l+s+s2}{\PYZdq{}}\PY{l+s+s2}{Neuron}\PY{l+s+s2}{\PYZdq{}}\PY{p}{)}

\PY{n}{ax} \PY{o}{=} \PY{n}{plt}\PY{o}{.}\PY{n}{subplot}\PY{p}{(}\PY{l+m+mi}{2}\PY{p}{,} \PY{l+m+mi}{2}\PY{p}{,} \PY{l+m+mi}{4}\PY{p}{)}
\PY{n}{rasterplot}\PY{p}{(}\PY{n}{sim}\PY{o}{.}\PY{n}{trange}\PY{p}{(}\PY{p}{)}\PY{p}{,} \PY{n}{sim}\PY{o}{.}\PY{n}{data}\PY{p}{[}\PY{n}{minusA\PYZus{}spikes}\PY{p}{]}\PY{p}{,} \PY{n}{ax}\PY{p}{)}
\PY{n}{plt}\PY{o}{.}\PY{n}{xlim}\PY{p}{(}\PY{l+m+mi}{0}\PY{p}{,} \PY{l+m+mi}{2}\PY{p}{)}
\PY{n}{plt}\PY{o}{.}\PY{n}{title}\PY{p}{(}\PY{l+s+s2}{\PYZdq{}}\PY{l+s+s2}{\PYZhy{}A}\PY{l+s+s2}{\PYZdq{}}\PY{p}{)}
\PY{n}{plt}\PY{o}{.}\PY{n}{xlabel}\PY{p}{(}\PY{l+s+s2}{\PYZdq{}}\PY{l+s+s2}{Time (s)}\PY{l+s+s2}{\PYZdq{}}\PY{p}{)}
\PY{n}{plt}\PY{o}{.}\PY{n}{ylabel}\PY{p}{(}\PY{l+s+s2}{\PYZdq{}}\PY{l+s+s2}{Neuron}\PY{l+s+s2}{\PYZdq{}}\PY{p}{)}
\PY{n}{hide\PYZus{}input}\PY{p}{(}\PY{p}{)}
\end{Verbatim}
\end{tcolorbox}

    
    \begin{verbatim}
HtmlProgressBar cannot be displayed. Please use the TerminalProgressBar. It can be enabled with `nengo.rc.set('progress', 'progress_bar', 'nengo.utils.progress.TerminalProgressBar')`.
    \end{verbatim}

    
    
    
    
    \begin{verbatim}
HtmlProgressBar cannot be displayed. Please use the TerminalProgressBar. It can be enabled with `nengo.rc.set('progress', 'progress_bar', 'nengo.utils.progress.TerminalProgressBar')`.
    \end{verbatim}

    
    
    
            \begin{tcolorbox}[breakable, boxrule=.5pt, size=fbox, pad at break*=1mm, opacityfill=0]
\prompt{Out}{outcolor}{21}{\hspace{3.5pt}}
\begin{Verbatim}[commandchars=\\\{\}]
<IPython.core.display.HTML object>
\end{Verbatim}
\end{tcolorbox}
        
    \begin{center}
    \adjustimage{max size={0.9\linewidth}{0.9\paperheight}}{intro_files/intro_32_5.png}
    \end{center}
    { \hspace*{\fill} \\}
    
    Nonlinear transformations involve solving for a new set of decoding
weights. Let us add a third population connected to the second one and
use it to compute \((-A)^2\).

    \begin{tcolorbox}[breakable, size=fbox, boxrule=1pt, pad at break*=1mm,colback=cellbackground, colframe=cellborder]
\prompt{In}{incolor}{22}{\hspace{4pt}}
\begin{Verbatim}[commandchars=\\\{\}]
\PY{k}{with} \PY{n}{model}\PY{p}{:}
    \PY{n}{A\PYZus{}squared} \PY{o}{=} \PY{n}{nengo}\PY{o}{.}\PY{n}{Ensemble}\PY{p}{(}\PY{l+m+mi}{30}\PY{p}{,} \PY{n}{dimensions}\PY{o}{=}\PY{l+m+mi}{1}\PY{p}{,} \PY{n}{max\PYZus{}rates}\PY{o}{=}\PY{n}{Uniform}\PY{p}{(}\PY{l+m+mi}{80}\PY{p}{,} \PY{l+m+mi}{100}\PY{p}{)}\PY{p}{)}
    \PY{n}{nengo}\PY{o}{.}\PY{n}{Connection}\PY{p}{(}\PY{n}{minusA}\PY{p}{,} \PY{n}{A\PYZus{}squared}\PY{p}{,} \PY{n}{function}\PY{o}{=}\PY{k}{lambda} \PY{n}{x}\PY{p}{:} \PY{n}{x} \PY{o}{*}\PY{o}{*} \PY{l+m+mi}{2}\PY{p}{)}
    \PY{n}{A\PYZus{}squared\PYZus{}spikes} \PY{o}{=} \PY{n}{nengo}\PY{o}{.}\PY{n}{Probe}\PY{p}{(}\PY{n}{A\PYZus{}squared}\PY{o}{.}\PY{n}{neurons}\PY{p}{)}
    \PY{n}{A\PYZus{}squared\PYZus{}probe} \PY{o}{=} \PY{n}{nengo}\PY{o}{.}\PY{n}{Probe}\PY{p}{(}\PY{n}{A\PYZus{}squared}\PY{p}{,} \PY{n}{synapse}\PY{o}{=}\PY{l+m+mf}{0.02}\PY{p}{)}
\end{Verbatim}
\end{tcolorbox}

    \begin{tcolorbox}[breakable, size=fbox, boxrule=1pt, pad at break*=1mm,colback=cellbackground, colframe=cellborder]
\prompt{In}{incolor}{23}{\hspace{4pt}}
\begin{Verbatim}[commandchars=\\\{\}]
\PY{k}{with} \PY{n}{nengo}\PY{o}{.}\PY{n}{Simulator}\PY{p}{(}\PY{n}{model}\PY{p}{)} \PY{k}{as} \PY{n}{sim}\PY{p}{:}
    \PY{n}{sim}\PY{o}{.}\PY{n}{run}\PY{p}{(}\PY{l+m+mi}{2}\PY{p}{)}

\PY{n}{plt}\PY{o}{.}\PY{n}{figure}\PY{p}{(}\PY{n}{figsize}\PY{o}{=}\PY{p}{(}\PY{l+m+mi}{10}\PY{p}{,} \PY{l+m+mf}{6.5}\PY{p}{)}\PY{p}{)}
\PY{n}{plt}\PY{o}{.}\PY{n}{subplot}\PY{p}{(}\PY{l+m+mi}{3}\PY{p}{,} \PY{l+m+mi}{2}\PY{p}{,} \PY{l+m+mi}{1}\PY{p}{)}
\PY{n}{plt}\PY{o}{.}\PY{n}{plot}\PY{p}{(}\PY{n}{sim}\PY{o}{.}\PY{n}{trange}\PY{p}{(}\PY{p}{)}\PY{p}{,} \PY{n}{sim}\PY{o}{.}\PY{n}{data}\PY{p}{[}\PY{n}{A\PYZus{}probe}\PY{p}{]}\PY{p}{)}
\PY{n}{plt}\PY{o}{.}\PY{n}{axhline}\PY{p}{(}\PY{l+m+mi}{0}\PY{p}{,} \PY{n}{color}\PY{o}{=}\PY{l+s+s1}{\PYZsq{}}\PY{l+s+s1}{k}\PY{l+s+s1}{\PYZsq{}}\PY{p}{)}
\PY{n}{plt}\PY{o}{.}\PY{n}{title}\PY{p}{(}\PY{l+s+s2}{\PYZdq{}}\PY{l+s+s2}{A}\PY{l+s+s2}{\PYZdq{}}\PY{p}{)}
\PY{n}{plt}\PY{o}{.}\PY{n}{xticks}\PY{p}{(}\PY{p}{(}\PY{p}{)}\PY{p}{)}
\PY{n}{plt}\PY{o}{.}\PY{n}{xlim}\PY{p}{(}\PY{l+m+mi}{0}\PY{p}{,} \PY{l+m+mi}{2}\PY{p}{)}

\PY{n}{plt}\PY{o}{.}\PY{n}{subplot}\PY{p}{(}\PY{l+m+mi}{3}\PY{p}{,} \PY{l+m+mi}{2}\PY{p}{,} \PY{l+m+mi}{3}\PY{p}{)}
\PY{n}{plt}\PY{o}{.}\PY{n}{plot}\PY{p}{(}\PY{n}{sim}\PY{o}{.}\PY{n}{trange}\PY{p}{(}\PY{p}{)}\PY{p}{,} \PY{n}{sim}\PY{o}{.}\PY{n}{data}\PY{p}{[}\PY{n}{minusA\PYZus{}probe}\PY{p}{]}\PY{p}{)}
\PY{n}{plt}\PY{o}{.}\PY{n}{axhline}\PY{p}{(}\PY{l+m+mi}{0}\PY{p}{,} \PY{n}{color}\PY{o}{=}\PY{l+s+s1}{\PYZsq{}}\PY{l+s+s1}{k}\PY{l+s+s1}{\PYZsq{}}\PY{p}{)}
\PY{n}{plt}\PY{o}{.}\PY{n}{title}\PY{p}{(}\PY{l+s+s2}{\PYZdq{}}\PY{l+s+s2}{\PYZhy{}A}\PY{l+s+s2}{\PYZdq{}}\PY{p}{)}
\PY{n}{plt}\PY{o}{.}\PY{n}{xticks}\PY{p}{(}\PY{p}{(}\PY{p}{)}\PY{p}{)}
\PY{n}{plt}\PY{o}{.}\PY{n}{xlim}\PY{p}{(}\PY{l+m+mi}{0}\PY{p}{,} \PY{l+m+mi}{2}\PY{p}{)}

\PY{n}{plt}\PY{o}{.}\PY{n}{subplot}\PY{p}{(}\PY{l+m+mi}{3}\PY{p}{,} \PY{l+m+mi}{2}\PY{p}{,} \PY{l+m+mi}{5}\PY{p}{)}
\PY{n}{plt}\PY{o}{.}\PY{n}{plot}\PY{p}{(}\PY{n}{sim}\PY{o}{.}\PY{n}{trange}\PY{p}{(}\PY{p}{)}\PY{p}{,} \PY{n}{sim}\PY{o}{.}\PY{n}{data}\PY{p}{[}\PY{n}{A\PYZus{}squared\PYZus{}probe}\PY{p}{]}\PY{p}{)}
\PY{n}{plt}\PY{o}{.}\PY{n}{axhline}\PY{p}{(}\PY{l+m+mi}{0}\PY{p}{,} \PY{n}{color}\PY{o}{=}\PY{l+s+s1}{\PYZsq{}}\PY{l+s+s1}{k}\PY{l+s+s1}{\PYZsq{}}\PY{p}{)}
\PY{n}{plt}\PY{o}{.}\PY{n}{title}\PY{p}{(}\PY{l+s+s2}{\PYZdq{}}\PY{l+s+s2}{(\PYZhy{}A)\PYZca{}2}\PY{l+s+s2}{\PYZdq{}}\PY{p}{)}
\PY{n}{plt}\PY{o}{.}\PY{n}{xlabel}\PY{p}{(}\PY{l+s+s2}{\PYZdq{}}\PY{l+s+s2}{Time (s)}\PY{l+s+s2}{\PYZdq{}}\PY{p}{)}
\PY{n}{plt}\PY{o}{.}\PY{n}{xlim}\PY{p}{(}\PY{l+m+mi}{0}\PY{p}{,} \PY{l+m+mi}{2}\PY{p}{)}

\PY{n}{ax} \PY{o}{=} \PY{n}{plt}\PY{o}{.}\PY{n}{subplot}\PY{p}{(}\PY{l+m+mi}{3}\PY{p}{,} \PY{l+m+mi}{2}\PY{p}{,} \PY{l+m+mi}{2}\PY{p}{)}
\PY{n}{rasterplot}\PY{p}{(}\PY{n}{sim}\PY{o}{.}\PY{n}{trange}\PY{p}{(}\PY{p}{)}\PY{p}{,} \PY{n}{sim}\PY{o}{.}\PY{n}{data}\PY{p}{[}\PY{n}{A\PYZus{}spikes}\PY{p}{]}\PY{p}{,} \PY{n}{ax}\PY{p}{)}
\PY{n}{plt}\PY{o}{.}\PY{n}{xlim}\PY{p}{(}\PY{l+m+mi}{0}\PY{p}{,} \PY{l+m+mi}{2}\PY{p}{)}
\PY{n}{plt}\PY{o}{.}\PY{n}{title}\PY{p}{(}\PY{l+s+s2}{\PYZdq{}}\PY{l+s+s2}{A}\PY{l+s+s2}{\PYZdq{}}\PY{p}{)}
\PY{n}{plt}\PY{o}{.}\PY{n}{xticks}\PY{p}{(}\PY{p}{(}\PY{p}{)}\PY{p}{)}
\PY{n}{plt}\PY{o}{.}\PY{n}{ylabel}\PY{p}{(}\PY{l+s+s2}{\PYZdq{}}\PY{l+s+s2}{Neuron}\PY{l+s+s2}{\PYZdq{}}\PY{p}{)}

\PY{n}{ax} \PY{o}{=} \PY{n}{plt}\PY{o}{.}\PY{n}{subplot}\PY{p}{(}\PY{l+m+mi}{3}\PY{p}{,} \PY{l+m+mi}{2}\PY{p}{,} \PY{l+m+mi}{4}\PY{p}{)}
\PY{n}{rasterplot}\PY{p}{(}\PY{n}{sim}\PY{o}{.}\PY{n}{trange}\PY{p}{(}\PY{p}{)}\PY{p}{,} \PY{n}{sim}\PY{o}{.}\PY{n}{data}\PY{p}{[}\PY{n}{minusA\PYZus{}spikes}\PY{p}{]}\PY{p}{,} \PY{n}{ax}\PY{p}{)}
\PY{n}{plt}\PY{o}{.}\PY{n}{xlim}\PY{p}{(}\PY{l+m+mi}{0}\PY{p}{,} \PY{l+m+mi}{2}\PY{p}{)}
\PY{n}{plt}\PY{o}{.}\PY{n}{title}\PY{p}{(}\PY{l+s+s2}{\PYZdq{}}\PY{l+s+s2}{\PYZhy{}A}\PY{l+s+s2}{\PYZdq{}}\PY{p}{)}
\PY{n}{plt}\PY{o}{.}\PY{n}{xticks}\PY{p}{(}\PY{p}{(}\PY{p}{)}\PY{p}{)}
\PY{n}{plt}\PY{o}{.}\PY{n}{ylabel}\PY{p}{(}\PY{l+s+s2}{\PYZdq{}}\PY{l+s+s2}{Neuron}\PY{l+s+s2}{\PYZdq{}}\PY{p}{)}

\PY{n}{ax} \PY{o}{=} \PY{n}{plt}\PY{o}{.}\PY{n}{subplot}\PY{p}{(}\PY{l+m+mi}{3}\PY{p}{,} \PY{l+m+mi}{2}\PY{p}{,} \PY{l+m+mi}{6}\PY{p}{)}
\PY{n}{rasterplot}\PY{p}{(}\PY{n}{sim}\PY{o}{.}\PY{n}{trange}\PY{p}{(}\PY{p}{)}\PY{p}{,} \PY{n}{sim}\PY{o}{.}\PY{n}{data}\PY{p}{[}\PY{n}{A\PYZus{}squared\PYZus{}spikes}\PY{p}{]}\PY{p}{,} \PY{n}{ax}\PY{p}{)}
\PY{n}{plt}\PY{o}{.}\PY{n}{xlim}\PY{p}{(}\PY{l+m+mi}{0}\PY{p}{,} \PY{l+m+mi}{2}\PY{p}{)}
\PY{n}{plt}\PY{o}{.}\PY{n}{title}\PY{p}{(}\PY{l+s+s2}{\PYZdq{}}\PY{l+s+s2}{(\PYZhy{}A)\PYZca{}2}\PY{l+s+s2}{\PYZdq{}}\PY{p}{)}
\PY{n}{plt}\PY{o}{.}\PY{n}{xlabel}\PY{p}{(}\PY{l+s+s2}{\PYZdq{}}\PY{l+s+s2}{Time (s)}\PY{l+s+s2}{\PYZdq{}}\PY{p}{)}
\PY{n}{plt}\PY{o}{.}\PY{n}{ylabel}\PY{p}{(}\PY{l+s+s2}{\PYZdq{}}\PY{l+s+s2}{Neuron}\PY{l+s+s2}{\PYZdq{}}\PY{p}{)}
\PY{n}{hide\PYZus{}input}\PY{p}{(}\PY{p}{)}
\end{Verbatim}
\end{tcolorbox}

    
    \begin{verbatim}
HtmlProgressBar cannot be displayed. Please use the TerminalProgressBar. It can be enabled with `nengo.rc.set('progress', 'progress_bar', 'nengo.utils.progress.TerminalProgressBar')`.
    \end{verbatim}

    
    
    
    
    \begin{verbatim}
HtmlProgressBar cannot be displayed. Please use the TerminalProgressBar. It can be enabled with `nengo.rc.set('progress', 'progress_bar', 'nengo.utils.progress.TerminalProgressBar')`.
    \end{verbatim}

    
    
    
            \begin{tcolorbox}[breakable, boxrule=.5pt, size=fbox, pad at break*=1mm, opacityfill=0]
\prompt{Out}{outcolor}{23}{\hspace{3.5pt}}
\begin{Verbatim}[commandchars=\\\{\}]
<IPython.core.display.HTML object>
\end{Verbatim}
\end{tcolorbox}
        
    \begin{center}
    \adjustimage{max size={0.9\linewidth}{0.9\paperheight}}{intro_files/intro_35_5.png}
    \end{center}
    { \hspace*{\fill} \\}
    
    \hypertarget{principle-3-dynamics}{%
\subsection{Principle 3: Dynamics}\label{principle-3-dynamics}}

So far, we have been considering the values represented by ensembles as
generic ``signals.'' However, if we think of them instead as state
variables in a dynamical system, then we can apply the methods of
control theory or dynamic systems theory to brain models. Nengo
automatically translates from standard dynamical systems descriptions to
descriptions consistent with neural dynamics.

In order to get interesting dynamics, we can connect populations
recurrently (i.e., to themselves).

Below is a simple harmonic oscillator implemented using this third
principle. It needs is a bit of input to get it started.

    \begin{tcolorbox}[breakable, size=fbox, boxrule=1pt, pad at break*=1mm,colback=cellbackground, colframe=cellborder]
\prompt{In}{incolor}{24}{\hspace{4pt}}
\begin{Verbatim}[commandchars=\\\{\}]
\PY{n}{model} \PY{o}{=} \PY{n}{nengo}\PY{o}{.}\PY{n}{Network}\PY{p}{(}\PY{n}{label}\PY{o}{=}\PY{l+s+s2}{\PYZdq{}}\PY{l+s+s2}{NEF summary}\PY{l+s+s2}{\PYZdq{}}\PY{p}{)}
\PY{k}{with} \PY{n}{model}\PY{p}{:}
    \PY{n+nb}{input} \PY{o}{=} \PY{n}{nengo}\PY{o}{.}\PY{n}{Node}\PY{p}{(}\PY{k}{lambda} \PY{n}{t}\PY{p}{:} \PY{p}{[}\PY{l+m+mi}{1}\PY{p}{,} \PY{l+m+mi}{0}\PY{p}{]} \PY{k}{if} \PY{n}{t} \PY{o}{\PYZlt{}} \PY{l+m+mf}{0.1} \PY{k}{else} \PY{p}{[}\PY{l+m+mi}{0}\PY{p}{,} \PY{l+m+mi}{0}\PY{p}{]}\PY{p}{)}
    \PY{n}{oscillator} \PY{o}{=} \PY{n}{nengo}\PY{o}{.}\PY{n}{Ensemble}\PY{p}{(}\PY{l+m+mi}{200}\PY{p}{,} \PY{n}{dimensions}\PY{o}{=}\PY{l+m+mi}{2}\PY{p}{)}
    \PY{n}{nengo}\PY{o}{.}\PY{n}{Connection}\PY{p}{(}\PY{n+nb}{input}\PY{p}{,} \PY{n}{oscillator}\PY{p}{)}
    \PY{n}{nengo}\PY{o}{.}\PY{n}{Connection}\PY{p}{(}
        \PY{n}{oscillator}\PY{p}{,} \PY{n}{oscillator}\PY{p}{,} \PY{n}{transform}\PY{o}{=}\PY{p}{[}\PY{p}{[}\PY{l+m+mi}{1}\PY{p}{,} \PY{l+m+mi}{1}\PY{p}{]}\PY{p}{,} \PY{p}{[}\PY{o}{\PYZhy{}}\PY{l+m+mi}{1}\PY{p}{,} \PY{l+m+mi}{1}\PY{p}{]}\PY{p}{]}\PY{p}{,} \PY{n}{synapse}\PY{o}{=}\PY{l+m+mf}{0.1}\PY{p}{)}
    \PY{n}{oscillator\PYZus{}probe} \PY{o}{=} \PY{n}{nengo}\PY{o}{.}\PY{n}{Probe}\PY{p}{(}\PY{n}{oscillator}\PY{p}{,} \PY{n}{synapse}\PY{o}{=}\PY{l+m+mf}{0.02}\PY{p}{)}
\end{Verbatim}
\end{tcolorbox}

    \begin{tcolorbox}[breakable, size=fbox, boxrule=1pt, pad at break*=1mm,colback=cellbackground, colframe=cellborder]
\prompt{In}{incolor}{25}{\hspace{4pt}}
\begin{Verbatim}[commandchars=\\\{\}]
\PY{k}{with} \PY{n}{nengo}\PY{o}{.}\PY{n}{Simulator}\PY{p}{(}\PY{n}{model}\PY{p}{)} \PY{k}{as} \PY{n}{sim}\PY{p}{:}
    \PY{n}{sim}\PY{o}{.}\PY{n}{run}\PY{p}{(}\PY{l+m+mi}{3}\PY{p}{)}

\PY{n}{plt}\PY{o}{.}\PY{n}{figure}\PY{p}{(}\PY{n}{figsize}\PY{o}{=}\PY{p}{(}\PY{l+m+mi}{10}\PY{p}{,} \PY{l+m+mf}{3.5}\PY{p}{)}\PY{p}{)}
\PY{n}{plt}\PY{o}{.}\PY{n}{subplot}\PY{p}{(}\PY{l+m+mi}{1}\PY{p}{,} \PY{l+m+mi}{2}\PY{p}{,} \PY{l+m+mi}{1}\PY{p}{)}
\PY{n}{plt}\PY{o}{.}\PY{n}{plot}\PY{p}{(}\PY{n}{sim}\PY{o}{.}\PY{n}{trange}\PY{p}{(}\PY{p}{)}\PY{p}{,} \PY{n}{sim}\PY{o}{.}\PY{n}{data}\PY{p}{[}\PY{n}{oscillator\PYZus{}probe}\PY{p}{]}\PY{p}{)}
\PY{n}{plt}\PY{o}{.}\PY{n}{ylim}\PY{p}{(}\PY{o}{\PYZhy{}}\PY{l+m+mf}{1.2}\PY{p}{,} \PY{l+m+mf}{1.2}\PY{p}{)}
\PY{n}{plt}\PY{o}{.}\PY{n}{xlabel}\PY{p}{(}\PY{l+s+s1}{\PYZsq{}}\PY{l+s+s1}{Time (s)}\PY{l+s+s1}{\PYZsq{}}\PY{p}{)}

\PY{n}{plt}\PY{o}{.}\PY{n}{subplot}\PY{p}{(}\PY{l+m+mi}{1}\PY{p}{,} \PY{l+m+mi}{2}\PY{p}{,} \PY{l+m+mi}{2}\PY{p}{)}
\PY{n}{plt}\PY{o}{.}\PY{n}{plot}\PY{p}{(}\PY{n}{sim}\PY{o}{.}\PY{n}{data}\PY{p}{[}\PY{n}{oscillator\PYZus{}probe}\PY{p}{]}\PY{p}{[}\PY{p}{:}\PY{p}{,} \PY{l+m+mi}{0}\PY{p}{]}\PY{p}{,} \PY{n}{sim}\PY{o}{.}\PY{n}{data}\PY{p}{[}\PY{n}{oscillator\PYZus{}probe}\PY{p}{]}\PY{p}{[}\PY{p}{:}\PY{p}{,} \PY{l+m+mi}{1}\PY{p}{]}\PY{p}{)}
\PY{n}{plt}\PY{o}{.}\PY{n}{grid}\PY{p}{(}\PY{p}{)}
\PY{n}{plt}\PY{o}{.}\PY{n}{axis}\PY{p}{(}\PY{p}{[}\PY{o}{\PYZhy{}}\PY{l+m+mf}{1.2}\PY{p}{,} \PY{l+m+mf}{1.2}\PY{p}{,} \PY{o}{\PYZhy{}}\PY{l+m+mf}{1.2}\PY{p}{,} \PY{l+m+mf}{1.2}\PY{p}{]}\PY{p}{)}
\PY{n}{plt}\PY{o}{.}\PY{n}{xlabel}\PY{p}{(}\PY{l+s+s1}{\PYZsq{}}\PY{l+s+s1}{\PYZdl{}x\PYZus{}1\PYZdl{}}\PY{l+s+s1}{\PYZsq{}}\PY{p}{)}
\PY{n}{plt}\PY{o}{.}\PY{n}{ylabel}\PY{p}{(}\PY{l+s+s1}{\PYZsq{}}\PY{l+s+s1}{\PYZdl{}x\PYZus{}2\PYZdl{}}\PY{l+s+s1}{\PYZsq{}}\PY{p}{)}
\PY{n}{hide\PYZus{}input}\PY{p}{(}\PY{p}{)}
\end{Verbatim}
\end{tcolorbox}

    
    \begin{verbatim}
HtmlProgressBar cannot be displayed. Please use the TerminalProgressBar. It can be enabled with `nengo.rc.set('progress', 'progress_bar', 'nengo.utils.progress.TerminalProgressBar')`.
    \end{verbatim}

    
    
    
    
    \begin{verbatim}
HtmlProgressBar cannot be displayed. Please use the TerminalProgressBar. It can be enabled with `nengo.rc.set('progress', 'progress_bar', 'nengo.utils.progress.TerminalProgressBar')`.
    \end{verbatim}

    
    
    
            \begin{tcolorbox}[breakable, boxrule=.5pt, size=fbox, pad at break*=1mm, opacityfill=0]
\prompt{Out}{outcolor}{25}{\hspace{3.5pt}}
\begin{Verbatim}[commandchars=\\\{\}]
<IPython.core.display.HTML object>
\end{Verbatim}
\end{tcolorbox}
        
    \begin{center}
    \adjustimage{max size={0.9\linewidth}{0.9\paperheight}}{intro_files/intro_38_5.png}
    \end{center}
    { \hspace*{\fill} \\}
    

    % Add a bibliography block to the postdoc
    
    
    
    \end{document}
